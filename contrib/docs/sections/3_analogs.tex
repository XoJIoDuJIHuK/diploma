\clearpage
\SectionWithSubsection{Постановка задачи и обзор аналогичных решений}
\SubsectionWithParagraph{Постановка задачи}

Веб-приложение должно позволять пользователю зарегистрироваться по адресу имени, электронной почты и паролю, подтвердить адрес электронной почты по ссылке из электронного письма, аутентифицироваться по адресу электронной почты и паролю, создать конфигурацию переводчика, загрузить исходную статью с введённым с клавиатуры текстом или при помощи загруженного текстового файла, выполнить перевод, используя созданную ранее конфигурацию переводчика, получить уведомление о завершении перевода, просмотреть переведённую статью. В случае, если перевод не удовлетворяет пользователя, веб-приложение должно позволять пользователю создать жалобу на перевод, а модератору – просмотреть жалобу и удовлетворить либо отклонить её.

\SubsectionBetweenParagraphs{Обзор аналогичных решений}


Одним из самых популярных сервисов по переводу текста с одного языка на другой является DeepL. Он предоставляет возможность перевода текста между различными языками, распознавание голоса, загрузку файлов и пересказ текста. Внешний вид страницы сервиса представлен на рисунке \ref{img:deepl}.

\begin{figure}[H]
    \centering
    \borderedimage[0.9\linewidth]{img/analog-1.jpg}{yes}
    \caption{Страница сервиса DeepL \label{img:deepl}}
\end{figure}
Данный сервис применяет нейронные сети для перевода текста, что положительно сказывается на качестве перевода текста.

В качестве второго аналогичного решения был рассмотрен сервис Google Translate. Внешний вид страницы данного сервиса представлен на странице \ref{img:gtranslate}.

\begin{figure}[H]
    \centering
    \borderedimage[0.85\linewidth]{img/analog-2.jpg}{yes}
    \caption{Страница сервиса Google Translate \label{img:gtranslate}}
\end{figure}
Это один из самых известных и широко используемых сервисов машинного перевода. Он также предоставляет возможность перевода текстовых файлов, а также более широкий выбор языков по сравнению с DeepL.

В качестве третьего аналогичного решения был рассмотрен сервис Wordvice. Внешний вид его страницы представлен на рисунке \ref{img:wordvice}.

\begin{figure}[H]
    \centering
    \borderedimage[0.9\linewidth]{img/analog-3.jpg}{yes}
    \caption{Страница сервиса Wordvice \label{img:wordvice}}
\end{figure}

Он также использует нейронные сети для перевода текста, предоставляет интеграцию с Microsoft Word и услуги обобщения и перефразирования текста при помощи искусственного интеллекта, а также поддерживает множество языков.

\SubsectionBetweenParagraphs{Выводы}

\begin{enumerate}
    \item Был рассмотрен сценарий использования веб-приложения, что позволяет выделить ключевые функции веб-приложения.
    \item Анализ существующих решений в области перевода текста выявил как преимущества, так и недостатки сервисов-конкурентов. Сервисы DeepL, Google Translate и Wordvice предоставляют базовый функционал, включая перевод текста, загрузку текстовых документов и выбор языков. Среди ключевых недостатков отмечается отсутствие персонализации, невозможность выбора средства выполнения перевода и его стиля, а также отсутствие жалоб на переводы.
\end{enumerate}
