\clearpage
\SectionWithSubsection{Руководство программиста}
\SubsectionWithParagraph{Настройка окружения}

Приложение разворачивалось на системе Ubuntu Server 24.04. Для корректной работы необходимо выполнить следующие шаги:

\begin{itemize}
    \item включить Uncomplicated Firewall при помощи команды “sudo ufw enable”;
    \item добавить перенаправление портов для доступа к веб-приложению при помощи команд “sudo ufw allow 80” и “sudo ufw allow 443”;
    \item опционально включить доступ по SSH при помощи команды “sudo ufw allow ssh” для доступа с удалённой машины;
    \item получить IP-адрес сервера при помощи команды “ip a”;
    \item занести полученный IP-адрес в файл hosts в формате “192.168.122.233 ugabuntu.com”;
    \item создать в домашнему каталоге серверного пользователя папку проекта веб-приложения, в которой будут находиться необходимые файлы, и перейти в неё при помощи команды “mkdir gptranslate \&\& cd gptranslate”;
    \item создать в папке все необходимые файлы, представленные в Приложении А и Приложении Г.
\end{itemize}

Для развёртывания веб-приложения применяется инструмент Docker Compose. Перед развёртыванием веб-приложения необходимо убедиться, что в системе установлены Docker Engine и Docker Compose при помощи команд docker version и docker compose version. В случае, если любая из указанных технологий не установлена, её необходимо установить согласно подходящей инструкции на официальном сайте, например, [13] для Docker Engine и [14] для Docker Compose.

Для корректного функционирования веб-приложения необходимо создать сеть Docker при помощи команды “docker network create a”. Данная сеть объединяет контейнеры в рамках Docker Compose и позволяет им коммуницировать между собой. Также данная сеть позволяет подключать к веб-приложению внешние сервисы, развёрнутые на локальной машине  в Docker, но не входящие в один проект Docker Compose с веб-приложением.

Для корректной работы веб-приложения ему необходим доступ к внешнему сервису g4f. Он может находиться в любом удобном месте: на локальной машине или на удалённом сервере. Для большего удобства можно развернуть его в Docker и добавить в ранее созданную сеть. Для этого нужно скачать базовый образ при помощи команды “docker pull hlohaus789/g4f:0.3.9.7”, развернуть его при помощи команды “docker run --detach --name g4f hlohaus789/g4f:0.3.9.7”, добавить созданный контейнер в сеть при помощи команды “docker network connect a g4f”.

\SubsectionBetweenParagraphs{Развёртывание приложения}

В папке веб-приложения необходимо создать файл .env, в котором нужно указать необходимые значения переменных окружения, используемых веб-приложением, таких как ключ доступа Unisender, логин и пароль для доступа к базе данных и так далее. Примеры объявления переменных окружения находится в файле .example.env. За адрес сервиса g4f отвечает переменная G4F\_ADDRESS. Ей необходимо присвоить адрес данного сервиса в формате “http://address:port”. В случае, если данный сервис был развёрнут на локальной машине в Docker согласно вышеуказанной инструкции, его адрес будет равен “http://g4f:1337”.

Далее в корневой папке веб-приложения необходимо последовательно выполнить команды “docker build -t diploma-base -f contrib/docker/base/Dockerfile .” и “docker compose --env-file=.env -f contrib/docker/docker-compose.prod.yaml up -d --build”. Эти команды создадут новую сеть Docker, соберут базовый образ для контейнеров из исходного кода и запустят все необходимые контейнеры соответственно. Проверить доступность сервиса g4f можно при помощи команды “docker exec -t docker-api-1 bash -c "/app/contrib/docker/wait-for-it.sh \textbackslash"g4f:1337\textbackslash" -t 30 -- echo \textbackslash"Сервис доступен\textbackslash""”.

В папке contrib/persistent\_data находятся .json файлы с данными, которыми будет заполнена база данных по умолчанию:

\begin{itemize}
    \item languages.json хранит информацию о доступных для перевода языках в формате словаря, чьими ключами являются названия языков, а значениями     \item их трёхбуквенные коды ISO 639-3:2007;
    \item models.json хранит массив массивов, хранящих отображаемое название модели и внутренние названия модели и провайдера, используемые для запросов к сервису g4f;
    \item prompts.json хранит массив массивов, хранящих название стиля перевода и текст стиля перевода;
    \item report-reasons.json хранит массив словарей с идентификатором, названием и позицией при сортировке.
\end{itemize}

При каждом запуске контейнера api будет производиться проверка на наличие данных, которых нет в базе данных, и отсутствующие строки будут добавлены автоматически.

Также при запуске контейнера api автоматически создаётся администратор с адресом электронной почты admin@d.com и паролем string и производится обновление структуры базы данных в соответствии с файлами миграций, находящихся в папке src/database/alembic/versions.

\SubsectionBetweenParagraphs{Проверка работоспособности приложения}

После развёртывания веб-приложения по адресу https://localhost будет доступна веб-страница веб-приложения. Также приложение должно быть доступно с других компьютеров в локальной сети по IP-адресу хоста. Шаги по проверке работоспособности развёрнутого веб-приложения описаны в разделе 5.

\SubsectionBetweenParagraphs{Выводы}

\begin{itemize}
    \item Было создано руководство, позволяющее развернуть и протестировать работоспособность веб-приложения в системе Ubuntu Server 24.04.
    \item Руководство также описывает локальное развёртывание сервиса g4f, используемого веб-приложением.
\end{itemize}
