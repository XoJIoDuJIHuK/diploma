\clearpage
\SectionWithSubsection{Тестирование веб-приложения}
\SubsectionWithParagraph{Функциональное тестирование}

Для тестирования работоспособности веб-приложения необходимо добавить объекты пользователя и модератора в таблицу Users (роли “user” и “moderator” соответственно; имя, пароль и адрес электронной почты произвольные). Объект администратора, а также языки, модели, стили перевода и причины для жалоб добавляются в базу автоматически при развёртывании веб-приложения в Docker Compose. Также автоматически создаётся база данных, и в ней создаются все необходимые объекты.

Для проверки функций приложения рекомендуется использовать инструмент OpenAPI. Данный инструмент генерирует документацию на основе исходного кода веб-приложения. Фреймворк FastAPI включает данный инструмент, и страница документации по умолчанию доступна по IP-адресу сервера, на котором развёрнуто приложение, и пути запроса /api/docs. Внешний вид данной страницы представлен в Приложении В. Для отправки запроса необходимо кликнуть по нужному элементу списка, нажать на кнопку “Try it out” ввести необходимые данные (тело запроса и его параметры) и нажать на кнопку “Execute”. Описание тестирования функций веб-приложения представлено в таблице \ref{tab:function_testing}.

\begin{longtable}{|
    p{33mm}|
    p{90mm}|
    p{42mm}|
}
    \caption[]{Описание тестирования функций веб-приложения \label{tab:function_testing}} \\ \hline
    \endfirsthead
    \multicolumn{3}{l}{Продолжение таблицы \thetable} \endhead
    Функция веб-приложения & Описание тестирования & Итог тестирования функции \\ \hline
    1 Изменение учётной записи & Аутентифицироваться в качестве пользователя, получить идентификатор своего пользователя при помощи GET запроса по адресу /api/users/me/, отправить POST запрос по пути  /api/users/\{идентификатор своего пользователя\}/name/, указав в теле запроса желаемое имя в параметре name (формат тела запроса – JSON). Сервер должен вернуть объект обновлённого пользователя в формате JSON & Работоспособность функции протестирована, ошибок не обнаружено \\ \hline
    2 Просмотр открытых сессий & Аутентифицироваться в качестве пользователя, отправить GET запрос на адрес /api/sessions/. Сервер должен вернуть список сессий в формате JSON & Работоспособность функции протестирована, ошибок не обнаружено \\ \hline
    3 Завершение открытых сессий & Аутентифицироваться в качестве пользователя, отправить POST запрос на адрес /api/sessions/close/. Сервер должен вернуть сообщение об успешном закрытии всех сессий & Работоспособность функции протестирована, ошибок не обнаружено \\ \hline
    4 Изменение списка исходных статей & Аутентифицироваться в качестве пользователя, отправить POST запрос на адрес /api/articles/, указав в теле запроса заголовок (title), текст (text) и идентификатор языка (language\_id) загружаемой статьи в формате JSON. Сервер должен вернуть объект статьи в формате JSON. Получить список языков в формате JSON можно, отправив GET запрос на адрес /api/languages/. Сервер должен вернуть список в формате JSON & Работоспособность функции протестирована, ошибок не обнаружено \\ \hline
    5 Изменение списка переведённых статей & Аутентифицироваться в качестве пользователя, отправить POST запрос на адрес /api/translation/, указав в теле запроса идентификатор статьи, которую нужно перевести (article\_id), список идентификаторов языков, на которые нужно перевести статью (target\_language\_ids), идентификатор стиля перевода (prompt\_id) и идентификатор модели перевода (model\_id). Сервер должен вернуть сообщение о запуске перевода, через некоторое время, зависящее от объёма статьи, в таблице Notifications должна появиться запись об успешном или неуспешном переводе статьи. Списки моделей и стилей перевода можно получить по GET запросам на адреса /api/models/ и /api/prompts/ соответственно & Работоспособность функции протестирована, ошибок не обнаружено \\ \hline
    6 Изменение списка жалоб на переводы своих статей & Аутентифицироваться в качестве пользователя, отправить запрос на адрес /api/articles/{идентификатор статьи}/report/, в теле запроса указать текст жалобы (text) и идентификатор причины жалобы (reason\_id). Сервер должен вернуть объект жалобы в формате JSON. Список доступных причин жалоб можно получить при помощи GET запроса на адрес /api/report-reasons/ & Работоспособность функции протестирована, ошибок не обнаружено \\ \hline
    7 Просмотр своих уведомлений & Аутентифицироваться в качестве пользователя, отправить GET запрос на адрес /api/notifications/. Сервер должен вернуть список непрочитанных уведомлений в формате JSON & Работоспособность функции протестирована, ошибок не обнаружено \\ \hline
    8 Изменение списка комментариев к жалобам на переводы своих статей & Протестировать функцию 10 Создание комментария, затем функцию 9 Получение списка комментариев к жалобе & Работоспособность функции протестирована, ошибок не обнаружено \\ \hline
    9 Получение списка комментариев к жалобе & Аутентифицироваться в качестве пользователя, отправить GET запрос на адрес /api/articles/ {идентификатор переведённой статьи, для жалобы на которую требуется получить список комментариев} /report/comments/. Сервер должен вернуть список комментариев в формате JSON & Работоспособность функции протестирована, ошибок не обнаружено \\ \hline
    10 Создание комментария & Аутентифицироваться в качестве пользователя, отправить POST запрос на адрес /api/articles/ {идентификатор переведённой статьи, для жалобы на которую требуется создать комментарий} /report/comments/, в запросе указать текст комментария (text). Сервер должен вернуть объект комментария в формате JSON & Работоспособность функции протестирована, ошибок не обнаружено \\ \hline
    11 Изменение списка настроек переводчика & Аутентифицироваться в качестве пользователя, отправить POST запрос на адрес /api/configs/, в запросе указать название конфигурации (name), идентификатор стиля перевода (prompt\_id), идентификатор модели перевода (model\_id) и список конечных языков (language\_ids). Сервер должен вернуть объект конфигурации в формате JSON & Работоспособность функции протестирована, ошибок не обнаружено \\ \hline
    12 Регистрация & Отправить POST запрос на адрес /api/auth/register/, указав в теле запроса имя пользователя (name), адрес электронной почты (email) и пароль (password). Сервер должен вернуть сообщение об успешной регистрации & Работоспособность функции протестирована, ошибок не обнаружено \\ \hline
    13 Аутентификация & Отправить POST запрос на адрес /api/auth/login/, указав в теле запроса адрес электронной почты (email) и пароль (password). Сервер должен вернуть сообщение об успешной аутентификации & Работоспособность функции протестирована, ошибок не обнаружено \\ \hline
    14 Изменение списка открытых жалоб & Аутентифицироваться в качестве модератора, отправить на адрес /api/articles/ {идентификатор статьи, жалобу на которую нужно изменить} /report/status/ POST запрос, указав в параметрах запроса новый статус жалобы (Отклонена или Удовлетворена). Сервер должен вернуть объект жалобы в формате JSON & Работоспособность функции протестирована, ошибок не обнаружено \\ \hline
    15 Создание комментариев для жалоб & Аутентифицироваться в качестве модератора, отправить на адрес /api/articles/ { идентификатор статьи, для жалобы на которую нужно создать комментарий} /report/comments/ POST запрос, указав в теле запроса текст комментария (text). Сервер должен вернуть объект комментария в формате JSON & Работоспособность функции протестирована, ошибок не обнаружено \\ \hline
    16 Просмотр статистики жалоб & Аутентифицироваться в качестве администратора, отправить GET запрос на адрес /api/analytics/models-stats/. Сервер должен вернуть данные по жалобам для каждой модели перевода в формате JSON & Работоспособность функции протестирована, ошибок не обнаружено \\ \hline
    17 Изменение списка стилей перевода & Аутентифицироваться в качестве администратора, отправить POST запрос на адрес /api/prompts/, в теле запроса указать название (title) и текст (text) стиля перевода Сервер должен вернуть объект стиля перевода в формате JSON & Работоспособность функции протестирована, ошибок не обнаружено \\ \hline
    18 Изменение списка моделей перевода & Аутентифицироваться в качестве администратора, отправить POST запрос на адрес /api/models/, в теле запроса указать отображаемое название (show\_name), название (name) и провайдер (provider) модели перевода Сервер должен вернуть объект модели перевода в формате JSON & Работоспособность функции протестирована, ошибок не обнаружено \\ \hline
    19 Изменение списка пользователей & Аутентифицироваться в качестве администратора, отправить POST запрос на адрес /api/users/, в теле запроса указать имя (name), адрес электронной почты (email), флаг, указывающий, подтверждена ли почта (email\_verified), роль (role) и пароль (password) пользователя. Сервер должен вернуть объект созданного пользователя в формате JSON & Работоспособность функции протестирована, ошибок не обнаружено \\ \hline
    20 Создание пользователей & Аналогично тестированию функции 19 Изменение списка пользователей, но роль в теле запроса должна быть “Пользователь” & Работоспособность функции протестирована, ошибок не обнаружено \\ \hline
    21 Создание модераторов & Аналогично тестированию функции 19 Изменение списка пользователей, но роль в теле запроса должна быть “Модератор” & Работоспособность функции протестирована, ошибок не обнаружено \\ \hline
    22 Создание администраторов & Аналогично тестированию функции 19 Изменение списка пользователей, но роль в теле запроса должна быть “Администратор” & Работоспособность функции протестирована, ошибок не обнаружено \\ \hline
\end{longtable}

Таким образом, были протестированы все ключевые функции веб-приложения, ошибок не обнаружено.

\SubsectionBetweenParagraphs{Нагрузочное тестирование}

Нагрузочное тестирование является критическим компонентом обеспечения надежности и производительности веб-приложений. Его целью является выявление определение максимальной пропускной способности приложения, идентификация предельных характеристик и количественная оценка производительности приложения в различных условиях.

В силу того, что веб-приложение не выполняет задач, требующих большого времени центрального процессора, основную часть времени обработки запроса занимает работа с базой данных. Для проверки поведения приложения под нагрузкой был разработан модуль tests.hot\_load.py, представленный в Приложении А. Данный модуль реализует класс HotLoad, предназначенный для отправки большого количества запросов на протяжении заданного времени. Для этого он использует класс Pool стандартного пакета multiprocessing. Использование нескольких процессов позволяет избежать ошибок отправки запросов из одного потока, при которых запросы не отправляются полностью.

Функции, предназначенные для запуска процессов в данном классе, представлена в листинге \ref{listing:hot_load_launch_functions}.

\begin{lstlisting}[caption={Функции запуска процессов класса HotLoad}, label={listing:hot_load_launch_functions}]
    def run_process(self, process_number: int, *args) -> int:
        worker_start_id = process_number * self.workers_number
        loop = asyncio.get_event_loop()
        result = loop.run_until_complete(self.run_workers(worker_start_id))
        return result

    async def run(self) -> float:
        if self.on_startup_callable:
            self.headers = await self.on_startup_callable()

        with multiprocessing.Pool(processes=self.processes_number) as pool:
            results = pool.map(self.run_process, range(self.processes_number))
        mean_rps = sum(results) / self.duration.total_seconds()

        if self.on_teardown_callable:
            await self.on_teardown_callable()

        return mean_rps
\end{lstlisting}

Для запуска теста необходимо выполнить команду “docker exec docker-api-1 bash -c “python tests/test\_six\_hot\_loads.py””. Данный тест выполняет повторяющиеся GET и POST запросы к серверу при помощи шести дочерних процессов на протяжении 30 секунд. По истечении заданного времени в терминал будет выведено среднее количество выполненных запросов в секунду.

Тестирование выявило некоторые ошибки в исходном коде. В частности, одна сессия базы данных использовалась в разных обработчиках конкурентно, что приводило к ошибкам. Ошибка была решена использованием примитива синхронизации Semaphore из стандартного пакета asyncio. Тестирование показало высокую пропускную способность приложения: порядка 80 запросов в секунду для запросов, получающих данные из базы данных и добавляющих данные в неё.

\SubsectionBetweenParagraphs{Выводы}

\begin{enumerate}
    \item Все функциональные возможности пользователей были протестированы, обнаруженные ошибки были исправлены.
    \item Веб-приложение было протестировано в условиях поступления большого количества запросов, по результатам которого показало высокую пропускную способность и устойчивость к нагрузкам.
\end{enumerate}
