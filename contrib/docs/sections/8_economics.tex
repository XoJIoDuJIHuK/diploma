\clearpage
\SectionWithSubsection{Технико-экономическое обоснование проекта}
\SubsectionWithParagraph{Общая характеристика разрабатываемого программного средства}

Основной целью экономического раздела является экономическое обоснование целесообразности разработки веб-приложения, представленного в дипломном проекте. В данном разделе проводится расчёт затрат на всех стадиях разработки и доходность продажи веб-приложения конечному заказчику.

Разработанное веб-приложение позволяет пользователям переводить значительные объёмы текста с одного языка на множество других языков. Для этого пользователям предоставлена возможность загружать статьи, создавать конфигурации перевода из языковых моделей и стилей перевода и запускать задачи на перевод с выбранной конфигурацией. Для обработки некачественных переводов присутствует механизм жалоб, позволяющий пользователям вернуть затраченные на перевод токены после рассмотрения жалобы на не понравившийся перевод модераторам. Администратору доступно управление моделями, стилями и пользователями, а также просмотр статистики.

Во время разработки дипломного проекта использовалась технология FastAPI для написания серверной части приложения и библиотека Vue.js для написания клиентской части приложения. Данное веб-приложение разработано для последующего использования в коммерческих целях.

\SubsectionBetweenParagraphs{Иходные данные для проведения расчётов}

Источниками исходных данных для последующих расчётов выступают действующие законы и нормативно-правовые акты. Исходные данные для расчёта стоимости программного продукта представлены в таблице \ref{tab:economics_params}.

\begin{longtable}{|p{85mm}|p{40mm}|p{40mm}|}
  \caption{Параметры, применяемые при расчёте стоимости разработки},
    \label{tab:economics_params} \\
    \hline
    Параметр & Условные обозначения & Норматив \\ \hline
    \endfirsthead
    \multicolumn{3}{l}{Продолжение таблицы \thetable} \endhead
    Норматив дополнительной заработной платы, \% & $\text{Н}_\text{дз}$ & 15 \\ \hline
    Ставка отчислений в Фонд социальной защиты населения, \% & $\text{Н}_\text{фсзн}$ & 34 \\ \hline
    Ставка отчислений в БРУСП "Белгосстрах", \% & $\text{Н}_\text{бгс}$ & 0.6 \\ \hline
    Норматив накладных расходов, \% & $\text{Н}_\text{обп, обх}$ & 50 \\ \hline
    Норматив расходов на сопровождение и адаптацию, \% & $\text{Н}_\text{рса}$ & 10 \\ \hline
    Ставка НДС, \% & $\text{Н}_\text{НДС}$ & 20 \\ \hline
    Налог на прибыль, \% & $\text{Н}_\text{п}$ & 20 \\ \hline
\end{longtable}

В ходе проведения маркетингового анализа была выявлена стоимость разработки программного продукта для перевода текста. Цены разработки аналогичных программных средств представлены в таблице \ref{tab:economics_analogs}.

\begin{longtable}{|p{30mm}|p{30mm}|p{30mm}|p{75mm}|}
  \caption{Маркетинговый анализ аналогов},
    \label{tab:economics_analogs} \\
    \hline
    Продукт-аналог & Источник & Стоимость, руб. & Примечание \\ \hline
    \endfirsthead
    \multicolumn{4}{l}{Продолжение таблицы \thetable} \endhead
    DeepL & \url{https://deepl.com} & 75000 & Данный аналог имеет множество дополнительных функций, связанных с нейронными сетями, что делает разработку дороже. Данные взяты с веб-сайта \url{https://majento.ru/index.php?page=seo-analize/site-cost} \\ \hline
  Google Translate & \url{https://translate.google.com} & 1000000 & Данный сервис был одним из первопроходчиков в области онлайн-перевода и постоянно развивается, что увеличивает стоимость разработки. Данные были взяты с веб-сайта \url{https://www.siteprice.org/website-worth/} translate.google.com \\ \hline
    Wordvice & \url{https://wordvice.ai} & 70000 & По аналогии с DeepL данный сервис предоставляет множество дополнительных функций и интеграций, что увеличивает стоимость разработки. Данные взяты с веб-сайта \url{https://majento.ru/index.php?page=seo-analize/site-cost} \\ \hline
\end{longtable}

В ходе проведения маркетингового анализа была определена стоимость разработки аналогичного программного продукта по переводу текста. Средняя цена разработки аналогичного продукта составляет 381000 рублей.

\SubsectionWithSubsubsection{Обоснование цены программного средства}

\SubsubsectionWithParagraph{Расчёт затрат рабочего времени на разработку программного средства}

Для определения затрат рабочего времени нужно учесть содержание работ, представленное в таблице \ref{tab:economics_time_spent}.

\begin{longtable}{|p{70mm}|p{60mm}|p{35mm}|}
  \caption{Затраты рабочего времени на разработку веб-приложения},
    \label{tab:economics_time_spent} \\
    \hline
    Содержание & Исполнитель & Трудозатраты, часов \\ \hline
    \endfirsthead
    \multicolumn{3}{l}{Продолжение таблицы \thetable} \\ \hline
    Содержание & Исполнитель & Трудозатраты, часов \\ \hline
    \endhead
    Дизайн приложения & Дизайнер & 40 \\ \hline
    Определение общих требований & Бизнес-аналитик & 8 \\ \hline
   \multirow{2}{\linewidth}{Определение функциональных требований} & Тимлид & 8 \\ \cline{2-3}
    & Бизнес-аналитик & 8 \\ \hline
    Определение нефункциональных требований & Бизнес-аналитик & 8 \\ \hline
    Проектирование приложения & Техлид & 8 \\ \hline
    \multirow{2}{\linewidth}{Проектирование базы данных} & Middle бэкенд-разработчик & 8 \\ \cline{2-3}
    & Junior бэкенд-разработчик & 8 \\ \hline
    \multirow{2}{\linewidth}{Разработка серверной части приложения} & Middle бэкенд-разработчик & 88 \\ \cline{2-3}
    & Junior бэкенд-разработчик & 88 \\ \hline
    \multirow{3}{\linewidth}{Разработка сайта} & Middle фронтенд-разработчик & 80 \\ \cline{2-3}
    & Junior фронтенд-разработчик & 80 \\ \cline{2-3}
    & Дизайнер & 80 \\ \hline
    \multirow{4}{\linewidth}{Тестирование серверной части приложения} & Middle бэкенд-разработчик & 20 \\ \cline{2-3}
    & Junior бэкенд-разработчик & 20 \\ \cline{2-3}
    & Middle тестировщик & 20 \\ \cline{2-3}
    & Junior тестировщик & 20 \\ \hline
    \multirow{4}{\linewidth}{Тестирование сайта} & Middle фронтенд-разработчик & 20 \\ \cline{2-3}
    & Junior фронтенд-разработчик & 20 \\ \cline{2-3}
    & Middle тестировщик & 20 \\ \cline{2-3}
    & Junior тестировщик & 20 \\ \hline
    \multirow{2}{\linewidth}{Разработка технической документации} & Middle бэкенд-разработчик & 24 \\ \cline{2-3}
    & Junior бэкенд-разработчик & 24 \\ \hline
    \multirow{2}{\linewidth}{Разработка руководства пользователя} & Middle фронтенд-разработчик & 24 \\ \cline{2-3}
    & Junior фронтенд-разработчик & 24 \\ \hline
\end{longtable}



На основе этих данных можно производить расчёт затрат на заработную плату.

\SubsubsectionBetweenParagraphs{Расчёт основной заработной платы}

Проект разрабатывался командой из бизнес-аналитика, технического лидера, дизайнера, а также junior и middle фронтенд и бекэнд разработчиков и тестировщиков на протяжении двух месяцев. Было проведено исследование рыночных зарплат данных специалистов. Зарплаты всех работников представлены в таблице \ref{tab:economics_salaries}.

\begin{longtable}{|p{60mm}|p{52mm}|p{53mm}|}
  \caption[]{Ставки оплаты работников},
    \label{tab:economics_salaries} \\ \hline
    Специалист & Месячная зарплата & Ставка в час, руб \\ \hline \endfirsthead
    \multicolumn{3}{l}{Продолжение таблицы \thetable} \\ \hline
    \endhead
    Дизайнер & 3360 & 20 \\ \hline
    Бизнес-аналитик & 3500 & 20,83 \\ \hline
    Техлид & 16800 & 100 \\ \hline
    Junior бэкенд-разработчик & 1680 & 10 \\ \hline
    Middle бэкенд-разработчик & 5040 & 30 \\ \hline
    Junior фронтенд-разработчик & 1680 & 10 \\ \hline
    Middle фронтенд-разработчик & 4704 & 28 \\ \hline
    Junior тестировщик & 1680 & 10 \\ \hline
    Middle тестировщик & 4032 & 24 \\ \hline
\end{longtable}

Основная заработная плата отдельного специалиста будет рассчитываться по формуле \ref{eq:based_salaries}.

\noindent
\begin{minipage}{1\linewidth}
\begin{equation}\label{eq:based_salaries}
  \text{С}_\text{оз} = \text{Т}_\text{раз}\,\cdot\,\text{С}_\text{зп}\text{,}
\end{equation}
\begin{itemize}[nosep, leftmargin=0pt, labelindent=0pt, itemsep=0pt, parsep=0pt]
  \item[] где $\text{С}_\text{оз}$ -- основная заработная плата, руб.;
  \item[] \hspace*{12.5mm}$\text{Т}_\text{раз}$ -- трудоемкость (чел./час.);
  \item[] \hspace*{12.5mm}$\text{С}_\text{зп}$ -- средняя часовая ставка руб./час.
\end{itemize}
\end{minipage}

\begin{gather*}
  C_{\text{оз диз}}   = 120\,\cdot\,20\ = 2400\,\text{руб.}\\
  C_{\text{оз ба}}    = 24\,\cdot\,20.83\ = 499.92\,\text{руб.}\\
  C_{\text{оз тех}}   = 16\,\cdot\,100\ = 1600\,\text{руб.}\\
  C_{\text{оз мбр}}   = 140\,\cdot\,10\ = 1400\,\text{руб.}\\
  C_{\text{оз бр}}    = 140\,\cdot\,30\ = 4200\,\text{руб.}\\
  C_{\text{оз мфр}}   = 124\,\cdot\,10\ = 1240\,\text{руб.}\\
  C_{\text{оз фр}}    = 124\,\cdot\,28\ = 3472\,\text{руб.}\\
  C_{\text{оз мтест}} = 40\,\cdot\,10\ = 400\,\text{руб.}\\
  C_{\text{оз тест}}  = 40\,\cdot\,24\ = 960\,\text{руб.}\\
  C_{\text{оз}}       = 2400 + 499.92 + 1600 + 1400 + 4200 + 1240 \\+ 3472 + 400 + 960 = 16171.92\,\text{руб.}
\end{gather*}

Полученные значения основной заработной платы будут использованы в дальнейших расчётах.

\SubsubsectionBetweenParagraphs{Расчёт дополнительной заработной платы}

Дополнительная заработная плата зависит от основной заработной платы и определяется по формуле \ref{eq:add_salaries}.

\noindent
\begin{minipage}{1\linewidth}
\begin{equation}\label{eq:add_salaries}
  \text{С}_\text{дз} = \text{С}_\text{оз}\,\cdot\,\text{Н}_\text{дз}\text{,}
\end{equation}
\begin{itemize}[nosep, leftmargin=0pt, labelindent=0pt, itemsep=0pt, parsep=0pt]
  \item[] где $\text{С}_\text{оз}$ -- основная заработная плата, руб.;
  \item[] \hspace*{12.5mm}$\text{Н}_\text{дз}$ -- норматив дополнительной заработной платы, \%.
\end{itemize}
\end{minipage}

\begin{gather*}
  C_{\text{дз}} = 16171.92\,\cdot\,0.15 = 2425.79\,\text{руб.}
\end{gather*}

Полученные данные будут использоваться в последующих расчётах.

\SubsubsectionBetweenParagraphs{Расчёт отчислений в Фонд социальной защиты населения и по обязательному страхованию}

Отчисления в Фонд социальной защиты населения (ФСЗН) и по обязательному страхованию от несчастных случаев на производстве и профессиональных заболеваний в БРУСП «Белгосстрах» определяются в соответствии с действующими законодательными актами по нормативу в процентном отношении к фонду основной и дополнительной зарплаты исполнителей.

Отчисления в Фонд социальной защиты населения вычисляются по формуле \ref{eq:fszn}.

\noindent
\begin{minipage}{1\linewidth}
\begin{equation}\label{eq:fszn}
  \text{С}_\text{фсзн} = (\text{С}_\text{оз}\,+\,\text{С}_\text{дз})\,\cdot\,\text{Н}_\text{фсзн}\text{,}
\end{equation}
\begin{itemize}[nosep, leftmargin=0pt, labelindent=0pt, itemsep=0pt, parsep=0pt]
  \item[] где $\text{С}_\text{оз}$ -- основная заработная плата, руб.;
  \item[] \hspace*{12.5mm}$\text{С}_\text{дз}$ -- дополнительная заработная плата на конкретное ПС, руб.;
  \item[] \hspace*{12.5mm}$\text{Н}_\text{фсзн}$ -- норматив отчислений в Фонд социальной защиты населения, \%.
\end{itemize}
\end{minipage}

Отчисления в БРУСП «Белгосстрах» вычисляются по формуле \ref{eq:bgs}.

\noindent
\begin{minipage}{1\linewidth}
\begin{equation}\label{eq:bgs}
  \text{С}_\text{бгс} = (\text{С}_\text{оз}\,+\,\text{С}_\text{дз})\,\cdot\,\text{Н}_\text{бгс}\text{,}
\end{equation}
\begin{itemize}[nosep, leftmargin=0pt, labelindent=0pt, itemsep=0pt, parsep=0pt]
  \item[] где $\text{Н}_\text{бгс}$ -- норматив отчислений в БРУСП "Белгосстрах" населения, \%.
\end{itemize}
\end{minipage}

\begin{gather*}
  C_{\text{фсзн}} = (16171.92\,+\,2425.79)\,\cdot\,0.34 = 6323.22\,\text{руб.}\\
  C_{\text{бгс}} = (16171.92\,+\,2425.79)\,\cdot\,0.006 = 111.59\,\text{руб.}
\end{gather*}

Таким образом, общие отчисления в БРУСП «Белгосстрах» составили 111.59
руб., а в фонд социальной защиты населения -- 6323.22 руб.

\SubsubsectionBetweenParagraphs{Расчёт суммы прочих прямых затрат}

Для разработки веб-приложения использовались инструменты и технологии со свободной лицензией, что позволило избежать излишних трат. Для разработки потребовалась подписка на сервис OpenAI ChatGPT Plus стоимостью 120 руб. и аренда низкоклассового облачного сервера для развёртывания приложения стоимостью 28 руб. общей продолжительностью два месяца. Таким образом, сумма прямых затрат составила 148 руб.

\SubsubsectionBetweenParagraphs{Расчёт суммы накладных расходов}

Сумма накладных расходов -- произведение основной заработной платы исполнителей на конкретное программное средство на норматив накладных расходов в целом по организации. Она расчитывается по формуле \ref{eq:add_expenses}.

\noindent
\begin{minipage}{1\linewidth}
\begin{equation}\label{eq:add_expenses}
  \text{С}_\text{обп, обх} = \text{С}_\text{оз}\,\cdot\,\text{Н}_\text{обп, обх}\text{.}
\end{equation}
\vspace{-10pt}
\end{minipage}

Сумма накладных расходов составит:

\begin{gather*}
  C_{\text{обп, обх}} = 16171.92\,\cdot\,0.5 = 6885.96\,\text{руб.}
\end{gather*}

Полученные данные будут применяться в последующих расчётах.

\SubsubsectionBetweenParagraphs{Сумма расходов на разработку программного средства и расчёт полной себестоимости}

Сумма расходов на разработку программного средства определяется как сумма основной и дополнительной заработных плат исполнителей на конкретное программное средство, отчислений на социальные нужды, суммы прочих прямых затрат и суммы накладных расходов по формуле \ref{eq:sum_expenses}.

\noindent
\begin{minipage}{1\linewidth}
\begin{equation}\label{eq:sum_expenses}
  \text{С}_\text{р} = \text{С}_\text{оз} + \text{С}_\text{дз} + \text{С}_\text{фсзн} + \text{С}_\text{бгс} + \text{С}_\text{пз} + \text{С}_\text{обп, обх}\text{.}
\end{equation}
\end{minipage}

\begin{gather*}
  \text{С}_\text{р} = 16171.92 + 2425.79 + 6323.22 + 111.59 + 148 + 6885.96 = 33266.47\,\text{руб.}
\end{gather*}

Сумма расходов на разработку программного средства была вычислена на основе данных, рассчитанных ранее в данном разделе, и составила 33 266.47 рублей.

Полная себестоимость определяется как сумма двух элементов: суммы расходов на разработку и суммы расходов на сопровождение и адаптацию. Для данного приложения расходы на сопровождение и адаптацию отсутствуют, поэтому в данном случае полная себестоимость равна расходам на разработку и составляет 33 266.47 рублей.

\SubsubsectionBetweenParagraphs{Определение цены}

Приложение разрабатывается на заказ и передаётся заказчику. Его цена определяется желаемой рентабельности 20\%. Прибыль от реализации программного средства вычисляется по формуле \ref{eq:income}.

\noindent
\begin{minipage}{1\linewidth}
\begin{equation}\label{eq:income}
  \text{П}_\text{пс} = \text{С}_\text{п}\,\cdot\,\text{У}_\text{рент}\text{,}
\end{equation}
\begin{itemize}[nosep, leftmargin=0pt, labelindent=0pt, itemsep=0pt, parsep=0pt]
  \item[] где $\text{У}_\text{рент}$ -- уровень рентабельности, \%.
  % в документе ещё было нужно повторить тут Сп, но оно уже было, поэтому я его убрал
\end{itemize}
\end{minipage}

Цена разработки программного средства без налогов находится по формуле \ref{eq:price}.

\noindent
\begin{minipage}{1\linewidth}
\begin{equation}\label{eq:price}
  \text{Ц}_\text{р} = \text{С}_\text{п}\,\cdot\,\text{П}_\text{пс}\text{.}
\end{equation}
\end{minipage}

Сумма налога на добавленную стоимость рассчитывается по формуле \ref{eq:nds}.

\noindent
\begin{minipage}{1\linewidth}
\begin{equation}\label{eq:nds}
  \text{НДС} = \text{Ц}_\text{р}\,\cdot\,\text{Н}_\text{НДС}\text{,}
\end{equation}
\begin{itemize}[nosep, leftmargin=0pt, labelindent=0pt, itemsep=0pt, parsep=0pt]
  \item[] где $\text{Ц}_\text{р}$ -- цена разработки программного средства, руб.;
  \item[] \hspace*{12.5mm}$\text{Н}_\text{НДС}$ -- ставка НДС, \%.
\end{itemize}
\end{minipage}

Планируемая отпускная цена с НДС вычисляется по формуле \ref{eq:price_with_nds}.

\noindent
\begin{minipage}{1\linewidth}
\begin{equation}\label{eq:price_with_nds}
  \text{Ц}_\text{с НДС} = \text{Ц}_\text{р}\,+\,\text{НДС}\text{.}
\end{equation}
\end{minipage}

Чистая прибыль за вычетом налога на прибыль вычисляется по формуле \ref{eq:clean_income}.

\noindent
\begin{minipage}{1\linewidth}
\begin{equation}{\label{eq:clean_income}}
    \text{П}_\text{чист} = \text{П}_\text{пс}\,\cdot\,(1\,-\,\text{Н}_\text{п})
\end{equation}
\end{minipage}

Исходя из вышеописанных данных рассчитаем прибыль от реализации программного средства, цену разработки без налогов, сумму налогов на добавленную стоимость, а также планируемую отпускную цену с учетом НДС.

\begin{gather*}
  \text{П}_\text{пс} = 33266.47\,\cdot\,0.2 = 6653.29\,\text{руб.}\\
  \text{Ц}_\text{р} = 33266.47\,+\,6653.29 = 39919.77\,\text{руб.}\\
  \text{НДС} = 39919.77\,\cdot\,0.2 = 7983.95\,\text{руб.}\\
  \text{Ц}_\text{с НДС} = 39919.77\,+\,7983.95 = 47903.72\,\text{руб.} \\
  \text{П}_\text{чист} = 6653.29\,\cdot\,(1\,-\,0.2)\,=\,5322,64\,{руб.}
\end{gather*}

Таким образом, цена разработанного приложения значительно ниже существующих аналогичных решений за счёт более узкой специализации и разработки малой командой, что является конкурентным преимуществом.

\SubsectionBetweenParagraphs{Выводы}

В данном разделе были осуществлены расчёты, представленные в таблице \ref{tab:econ_result}.

\begin{longtable}{|p{82mm}|p{82mm}|}
  \caption[]{Результаты расчётов},
    \label{tab:econ_result} \\ \hline
    \endfirsthead
    \multicolumn{2}{l}{Продолжение таблицы \thetable} \endhead
    Наименование показателя & Значение \\ \hline
    Время разработки, ч. & 768 \\ \hline
    Основная заработная плата, руб. & 16171.92 \\ \hline
    Дополнительная заработная плата, руб. & 2425.79 \\ \hline
    Отчисления в Фонд социальной защиты населения, руб. & 6323.22 \\ \hline
    Отчисления в БРУСП «Белгосстрах», руб. & 111.59 \\ \hline
    Прочие прямые затраты, руб. & 148 \\ \hline
    Накладные расходы, руб. & 6885.96 \\ \hline
    Полная себестоимость, руб. & 33266.47 \\ \hline
    Цена продукта, руб. & 47903.72 \\ \hline
    Прибыль от продажи, руб. & 6653.29 \\ \hline
    Чистая прибыль, руб. & 5322.64 \\ \hline
\end{longtable}

Таким образом, приложение было разработано за два месяца, полная себестоимость разработки составила 33 266.47 руб.

Приложение было разработано вследствие отсутствия на рынке решений, отвечающих специфическим требованиям, таким как прозрачность выбора инструмента перевода, персонализация и структурированное хранение переводов.
