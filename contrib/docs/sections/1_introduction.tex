\clearpage
\section*{Введение}
\addcontentsline{toc}{section}{Введение}

Тема проекта «Веб-приложение «GPTranslate» для перевода текста» означает, что результатом выполнения проекта является веб-приложение, позволяющее пользователям переводить текст на исходном языке в текст на другом языке (например, с польского языка на немецкий) с использованием внешнего сервиса «g4f», предоставляющего доступ к нейронным сетям. Доступ к внешнему сервису осуществляется по его API. Ограниченный объём текста означает, что в веб-приложении установлено ограничение на максимальную длину исходного текста.

Основная цель проекта – повысить эффективность и увеличить скорость перевода текста за счёт использования внешних сервисов, а также предоставить пользователям механизм обратной связи для улучшения сервиса.

Для достижения поставленной цели в рамках дипломного проекта были сформулированы следующие задачи: в разделе 1 провести анализ литературных источников; в разделе 2 провести анализ существующих сервисов перевода и выявить их преимущества и недостатки; в разделе 3 разработать архитектуру и структуру приложения, включая выбор технологий и структуру базы данных; в разделе 4 описать программную разработку с реализацией ключевых функций; в разделе 5 провести тестирование функционала и производительности; в разделе 6 подготовить техническую документацию и инструкции по развёртыванию приложения; в разделе 7 провести технико-экономический анализ проекта.

Целевая аудитория приложения включает широкий спектр пользователей: от профессиональных переводчиков и сотрудников международных компаний до владельцев веб-сайтов и блогеров, нуждающихся в качественном и быстром переводе своих материалов.

Для реализации веб-приложения было решено использовать язык программирования Python и фреймворк FastAPI из-за высокой гибкости, распространённости и простоты данных технологий.
