\clearpage
\SectionWithSubsection{Обзор литературных источников}
\SubsectionWithParagraph{Документация Python}

Официальная документация Python представляет собой комплексную систему материалов, организованную по принципу "от простого к сложному". Она выделяется среди других языков своей полнотой и доступностью изложения. Она включает в себя следующие ключевые разделы: учебник для начинающих (пошаговое введение в синтаксис и концепции языка, построенное по принципу "сценарной документации", что позволяет новичкам быстро добиваться нужных результатов по заранее расписанным шагам), справочник по библиотеке (исчерпывающее описание стандартной библиотеки, включающее подробные примеры использования каждого модуля), справочник по языку (детальное описание синтаксиса и семантики языка, включая формальные определения и спецификации), а также руководство по расширению и внедрению Python (информация для разработчиков, создающих модули на языках программирования C/C++ или встраивающих интерпретатор Python в другие приложения).

Документация Python решает следующие проблемы: обеспечение единого понимания языка среди разработчиков, стандартизация использования встроенных возможностей и библиотек, снижение порога вхождения для новых разработчиков, выступление в качестве авторитетного источника при разрешении технических споров и неоднозначностей.

\SubsectionBetweenParagraphs{Документация FastAPI}

Документация фреймворка FastAPI предоставляет полную информацию для использования данной технологии. Она содержит руководство пользователя для быстрого начала работы, пошаговые инструкции к различным аспектам фреймворка с пояснениями каждого шага и примерами для различных версий смежных технологий (например, с использованием typing.List в Python 3.8 и list в Python 3.9 и позднее), а также описание взаимодействия с прочими технологиями (например, Starlette и Pydantic).

Кроме того, документация предоставляет доступ к описанию типов, классов и функций с их подробным описанием, что позволяет ускорить разработку приложений. Также документация предоставляет рекомендации по развёртыванию приложений в продуктовых окружениях и настройке в случае нахождения за прокси.

\SubsectionBetweenParagraphs{Документация Pydantic}

Документация библиотеки валидации Pydantic организована как интерактивное руководство, сочетающее теоретические объяснения с практическими примерами. Она выделяется среди других библиотек своим подходом "проблема - решение", где каждый раздел начинается с описания конкретной задачи разработки, за чем следует описание её решения.

Так же, как документация FastAPI, документация Pydantic содержит руководство для начинающих для быстрого начала работы, руководства по основным аспектам библиотеки, а также раздел с описанием функций и классов, реализуемых библиотекой. Документация позволяет узнать, каким образом проектировать и реализовывать надёжные API с сериализацией и валидацией.

\SubsectionBetweenParagraphs{Документация PostgreSQL}

Документация СУБД PostgreSQL представляет исчерпывающее руководство по ключевым аспектам применения данной технологии, а также инструкции по развёртыванию и использованию. Кроме того, документация содержит полное описание диалекта PostgreSQL с примерами использования, что позволяет лучше изучить его для полноценного использования СУБД.

Кроме того, документация описывает внутренне устройство СУБД, включая индексы, системы обеспечения согласованности, доступности, репликации, партиционирования и шардирования, и заметки к выпуску каждой новой версии, включая мелкие и крупные изменения.

Документация PostgreSQL предоставляет исчерпывающую информацию о возможностях системы, облегчает процесс настройки и оптимизации производительности, помогает в проектировании схемы данных и разработке сложных запросов, а также служит справочником при решении проблем и отладке ошибок.
