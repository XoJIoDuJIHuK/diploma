\clearpage
\section*{Приложение А}

Исходный код наиболее значимых частей веб-приложения

\begin{lstlisting}
//src.routers.analytics.views.py
@router.get(
    '/models-stats/'
)
async def get_models_stats(
        user_info: UserInfo = Depends(JWTCookie(roles=[Role.admin])),
        db_session: AsyncSession = Depends(get_session)
):
    return await AnalyticsRepo.get_models_stats(db_session)


@router.get(
    '/prompts-stats/'
)
async def get_prompts_stats(
        user_info: UserInfo = Depends(JWTCookie(roles=[Role.admin])),
        db_session: AsyncSession = Depends(get_session)
):
    return await AnalyticsRepo.get_prompts_stats(db_session)




//src.routers.articles.views.py
@router.post(
    '/',
    response_model=DataResponse.single_by_key(
        'article',
        ArticleOutScheme
    ),
    responses=get_responses(400, 401, 403, 500)
)
async def upload_article(
        article_data: UploadArticleScheme,
        user_info: UserInfo = Depends(JWTCookie(roles=[Role.user])),
        db_session: AsyncSession = Depends(get_session)
):
    article = await ArticleRepo.create(
        article_data=CreateArticleScheme(
            title=article_data.title,
            text=article_data.text,
            language_id=article_data.language_id,
            user_id=user_info.id
        ),
        db_session=db_session
    )
    return DataResponse(
        data={
            'article': ArticleOutScheme.model_validate(article)
        }
    )


@router.put(
    '/{article_id}/',
    response_model=DataResponse.single_by_key(
        'article',
        ArticleOutScheme
    ),
    responses=get_responses(400, 401, 403, 404, 500)
)
async def update_article(
        new_article_data: EditArticleScheme,
        article_id: uuid.UUID = Path(),
        user_info: UserInfo = Depends(JWTCookie(roles=[Role.user])),
        db_session: AsyncSession = Depends(get_session),
):
    article = await ArticleRepo.get_by_id(article_id, db_session)
    if (
            not article
            or article.user_id != user_info.id
            or article.original_article_id is not None
    ):
        raise article_not_found_error
    if new_article_data.title is not None:
        article.title = new_article_data.title
    if new_article_data.text is not None:
        article.text = new_article_data.text
    db_session.add(article)
    await db_session.commit()
    await db_session.refresh(article)
    return DataResponse(
        data={
            'article': ArticleOutScheme.model_validate(article)
        }
    )


@router.delete(
    '/{article_id}/',
    response_model=DataResponse.single_by_key(
        'article',
        ArticleOutScheme
    ),
    responses=get_responses(400, 401, 403, 404, 500)
)
async def delete_article(
        article_id: uuid.UUID = Path(),
        user_info: UserInfo = Depends(JWTCookie(roles=[Role.user])),
        db_session: AsyncSession = Depends(get_session),
):
    article = await ArticleRepo.get_by_id(article_id, db_session)
    if not article or article.user_id != user_info.id:
        raise article_not_found_error
    article = await ArticleRepo.delete(
        article=article,
        db_session=db_session
    )
    return DataResponse(
        data={
            'article': ArticleOutScheme.model_validate(article)
        }
    )




//src.routers.auth.views.py
@router.post(
    '/login/',
    responses=get_responses(404)
)
async def login(
        login_data: LoginScheme,
        request: Request,
        db_session: AsyncSession = Depends(get_session)
):
    user = await UserRepo.get_by_email(
        email=login_data.email,
        db_session=db_session
    )
    if (
        not user or
        user.password_hash != get_password_hash(login_data.password)
    ):
        raise HTTPException(
            status_code=status.HTTP_404_NOT_FOUND,
            detail='Неправильные данные для входа'
        )
    if not user.email_verified:
        raise HTTPException(
            status_code=status.HTTP_400_BAD_REQUEST,
            detail='Подтвердите адрес электронной почты'
        )
    if AppConfig.close_sessions_on_same_device_login:
        await SessionRepo.close_all(
            user_id=user.id,
            ip=request.headers.get('X-Forwarded-For'),
            user_agent=get_user_agent(request),
            db_session=db_session
        )
    await db_session.refresh(user)
    tokens = await AuthHandler.login(
        user=user,
        request=request,
        db_session=db_session
    )
    response = JSONResponse({'detail': 'Аутентифицирован'})
    return get_authenticated_response(response, tokens)


@router.post(
    '/register/',
    response_model=BaseResponse,
    responses=get_responses(409)
)
async def register(
        registration_data: RegistrationScheme,
        db_session: AsyncSession = Depends(get_session)
):
    if await UserRepo.name_is_taken(
        name=registration_data.name,
        db_session=db_session
    ):
        raise HTTPException(
            status_code=status.HTTP_409_CONFLICT,
            detail='Имя занято'
        )
    user = await UserRepo.create(
        user_data=CreateUserScheme(
            name=registration_data.name,
            email=registration_data.email,
            email_verified=False,
            password=registration_data.password,
            role=Role.user
        ),
        db_session=db_session
    )
    await send_email_confirmation_message(
        user=user,
        email=registration_data.email,
        db_session=db_session
    )
    return BaseResponse(message='Регистрация успешна. Проверьте почту')


@router.post(
    '/restore-password/request/',
    response_model=BaseResponse,
    responses=get_responses(404)
)
async def request_password_restoration_code(
        email: EmailStr,
        db_session: AsyncSession = Depends(get_session)
):
    user = await UserRepo.get_by_email(
        email=email,
        db_session=db_session
    )
    if not user:
        raise HTTPException(
            status_code=status.HTTP_404_NOT_FOUND,
            detail='Неправильный адрес электронной почты'
        )
    confirmation_code = await ConfirmationCodeRepo.create(
        user_id=user.id,
        reason=ConfirmationType.password_reset,
        db_session=db_session
    )
    producer = KafkaProducer(
        bootstrap_servers=KafkaConfig.address,
        topic=KafkaConfig.mail_topic
    )
    kafka_message = SendEmailScheme(
        to_address=email,
        from_address=UnisenderConfig.from_address,
        from_name=UnisenderConfig.from_name,
        subject=UnisenderConfig.password_recovery_subject,
        template_id=UnisenderConfig.password_recovery_template_id,
        params={
            'link': f'{FrontConfig.address}'
                    f'{FrontConfig.change_password_endpoint}'
                    f'?code={confirmation_code.code}'
        }
    )
    await producer.send_message(kafka_message.model_dump(mode='json'))
    return BaseResponse(message='Сообщение отправляется на почту')


@router.patch(
    '/restore-password/confirm/',
    response_model=BaseResponse,
    responses=get_responses(400, 404)
)
async def restore_password(
        request_data: ResetPasswordScheme,
        db_session: AsyncSession = Depends(get_session)
):
    confirmation_code = await ConfirmationCodeRepo.get(
        value=request_data.code,
        reason=ConfirmationType.password_reset,
        db_session=db_session
    )
    if not confirmation_code:
        raise HTTPException(
            status_code=status.HTTP_404_NOT_FOUND,
            detail='Код восстановления пароля не найден'
        )
    new_password_hash = get_password_hash(request_data.new_password)
    await UserRepo.update_password_hash(
        user_id=confirmation_code.user_id,
        new_password_hash=new_password_hash,
        db_session=db_session
    )
    await ConfirmationCodeRepo.mark_as_used(
        confirmation_code=confirmation_code,
        db_session=db_session
    )
    return BaseResponse(message='Пароль успешно изменён')




//src.routers.config.views.py
@router.post(
    '/',
    response_model=DataResponse.single_by_key(
        'config',
        ConfigOutScheme
    ),
    responses=get_responses(400, 401, 409)
)
async def create_config(
        request: Request,
        config_data: CreateConfigScheme,
        db_session: AsyncSession = Depends(get_session),
        user_info: UserInfo = Depends(JWTCookie(roles=[Role.user]))
):
    config = await ConfigRepo.create(
        config_data=config_data,
        user_id=user_info.id,
        db_session=db_session
    )
    return DataResponse(
        data={
            'config': ConfigOutScheme.model_validate(config)
        }
    )


@router.put(
    '/{config_id}/',
    response_model=DataResponse.single_by_key(
        'config',
        ConfigOutScheme
    ),
    responses=get_responses(400, 401, 404, 409)
)
async def update_config(
        config_data: EditConfigScheme,
        config: TranslationConfig = Depends(get_config),
        db_session: AsyncSession = Depends(get_session),
        user_info: UserInfo = Depends(JWTCookie(roles=[Role.user]))
):
    config = await ConfigRepo.update(
        config=config,
        new_data=config_data,
        db_session=db_session
    )
    return DataResponse(
        data={
            'config': ConfigOutScheme.model_validate(config)
        }
    )


@router.delete(
    '/{config_id}/',
    response_model=BaseResponse,
    responses=get_responses(400, 401, 404, 409)
)
async def delete_config(
        request: Request,
        config: TranslationConfig = Depends(get_config),
        db_session: AsyncSession = Depends(get_session),
        user_info: UserInfo = Depends(JWTCookie(roles=[Role.user]))
):
    logger.info(f'Worker {request.headers.get('X-Worker-ID', 'unknown')} is trying to delete config {config.name[-1]}')
    config_name = config.name
    await ConfigRepo.delete(
        config=config,
        db_session=db_session
    )
    return BaseResponse(message=f'Конфиг {config_name} удалён')




//src.routers.models.views.py
@router.post(
    '/',
    response_model=DataResponse.single_by_key(
        'model',
        ModelOutScheme
    ),
    responses=get_responses(400, 401, 403, 409)
)
async def create_model(
        model_data: ModelCreateScheme,
        db_session: AsyncSession = Depends(get_session),
        user_info: UserInfo = Depends(JWTCookie(roles=[Role.admin]))
):
    await check_model_conflicts(
        model_data=model_data,
        existing_model_id=None,
        db_session=db_session
    )
    model = await ModelRepo.create(
        model_data=model_data,
        db_session=db_session
    )
    return DataResponse(
        data={
            'model': ModelOutScheme.model_validate(model)
        }
    )


@router.put(
    '/{model_id}/',
    response_model=DataResponse.single_by_key(
        'model',
        ModelOutScheme
    ),
    responses=get_responses(400, 401, 403, 404, 409)
)
async def update_model(
        model_data: ModelUpdateScheme,
        model_id: int = Path(),
        db_session: AsyncSession = Depends(get_session),
        user_info: UserInfo = Depends(JWTCookie(roles=[Role.admin]))
):
    model = await ModelRepo.get_by_id(
        model_id=model_id,
        db_session=db_session
    )
    if not model:
        raise HTTPException(
            status_code=status.HTTP_404_NOT_FOUND,
            detail='Модели не существует'
        )
    await check_model_conflicts(
        model_data=model_data,
        existing_model_id=model_id,
        db_session=db_session
    )
    model = await ModelRepo.update(
        model=model,
        new_model_data=model_data,
        db_session=db_session
    )
    return DataResponse(
        data={
            'model': ModelOutScheme.model_validate(model)
        }
    )


@router.delete(
    '/{model_id}/',
    response_model=BaseResponse,
    responses=get_responses(400, 401, 403, 404)
)
async def delete_model(
        model_id: int = Path(),
        db_session: AsyncSession = Depends(get_session),
        user_info: UserInfo = Depends(JWTCookie(roles=[Role.admin]))
):
    result = await ModelRepo.delete(
        model_id=model_id,
        db_session=db_session
    )
    return BaseResponse(message=result)




//src.routers.oauth.views.py
@router.get(
    '/login/'
)
async def redirect_to_provider(
        request: Request,
        provider: OAuthProvider,
):
    provider_authorize = get_oauth_provider(
        provider=provider,
        storage=RedisHandler()
    )
    new_session_data = {
        OAuthConfig.session_data_property: {
            'ip': request.headers.get('X-Forwarded-For'),
        }
    }
    request.session.update(new_session_data)
    authorization_url = await provider_authorize.get_auth_url()
    return RedirectResponse(authorization_url)


@router.get(
    '/{provider}/callback',
    summary='Validates auth code from provider and returns user\'s tokens',
    response_model=None
)
async def callback(
    request: Request,
    provider: OAuthProvider = Path(),
    db_session: AsyncSession = Depends(get_session),
):
    oauth_login_data = request.session.get(
        OAuthConfig.session_data_property
    )
    if not oauth_login_data:
        error_message = (
            f'Ошибка валидации сессии: {request.session}, отсутствует свойство'
            f' \'{OAuthConfig.session_data_property}\''
        )
        logger.error(error_message)
        raise Exception(error_message)


    provider_authorize = get_oauth_provider(
        provider=provider,
        storage=RedisHandler()
    )
    auth_token = await provider_authorize.callback(
        request=request
    )


    user_data = await provider_authorize.get_user_info(auth_token)
    logger.error(user_data)
    user_id = user_data.id
    provider_user_id = (str(user_id) if user_id else None)


    if email := user_data.email:
        user = await UserRepo.get_by_email(
            email=email,
            db_session=db_session
        )
        if not user:
            user = await UserRepo.register_for_oauth(
                role=Role.user,
                db_session=db_session,
                email=email,
                name=user_data.name,
                oauth_provider=provider,
                provider_id=provider_user_id,
            )
    else:
        user = await UserRepo.get_by_oauth_data(
            provider=provider,
            provider_id=provider_user_id,
            db_session=db_session
        )
        if not user:
            user = await UserRepo.register_for_oauth(
                email=None,
                name=user_data.name,
                role=Role.user,
                db_session=db_session,
                oauth_provider=provider,
                provider_id=provider_user_id,
            )
    db_session.add(user)
    await db_session.commit()
    await db_session.refresh(user)
    tokens = await AuthHandler.login(
        user=user,
        request=request,
        db_session=db_session
    )
    response = RedirectResponse(f'/')
    return get_authenticated_response(response, tokens)




//src.routers.prompts.views.py
@router.post(
    '/',
    response_model=DataResponse.single_by_key(
        'prompt',
        PromptOutScheme
    )
)
async def create_prompt(
        prompt_data: CreatePromptScheme,
        db_session: AsyncSession = Depends(get_session),
        user_info: UserInfo = Depends(JWTCookie(roles=[Role.admin]))
):
    prompt = await PromptRepo.create(
        prompt_data=prompt_data,
        db_session=db_session
    )
    return DataResponse(
        data={
            'prompt': PromptOutScheme.model_validate(prompt)
        }
    )


@router.put(
    '/{prompt_id}/',
    response_model=DataResponse.single_by_key(
        'prompt',
        PromptOutScheme
    )
)
async def update_prompt(
        prompt_data: EditPromptScheme,
        prompt: StylePrompt = Depends(get_prompt),
        db_session: AsyncSession = Depends(get_session),
        user_info: UserInfo = Depends(JWTCookie(roles=[Role.admin]))
):
    prompt = await PromptRepo.update(
        prompt=prompt,
        prompt_data=prompt_data,
        db_session=db_session
    )
    return DataResponse(
        data={
            'prompt': PromptOutScheme.model_validate(prompt)
        }
    )


@router.delete(
    '/{prompt_id}/',
    response_model=BaseResponse
)
async def delete_prompt(
        prompt: StylePrompt = Depends(get_prompt),
        db_session: AsyncSession = Depends(get_session),
        user_info: UserInfo = Depends(JWTCookie(roles=[Role.admin]))
):
    await PromptRepo.delete(
        prompt=prompt,
        db_session=db_session
    )
    return BaseResponse(message='Промпт удалён')




//src.routers.sessions.views.py
@router.get(
    '/',
    response_model=ListResponse[SessionOutScheme],
    responses=get_responses(400, 401)
)
async def get_sessions(
        user_info: UserInfo = Depends(JWTCookie()),
        db_session: AsyncSession = Depends(get_session),
        pagination: PaginationParams = Depends(get_pagination_params)
):
    sessions, count = await SessionRepo.get_list(
        user_id=user_info.id,
        pagination_params=pagination,
        db_session=db_session
    )
    return ListResponse[SessionOutScheme].from_list(
        items=sessions,
        total_count=count,
        params=pagination
    )


@router.post(
    '/close/',
    response_model=BaseResponse,
    responses=get_responses(400, 401)
)
async def close_sessions(
        user_info: UserInfo = Depends(JWTCookie()),
        db_session: AsyncSession = Depends(get_session),
):
    refresh_token_ids = await SessionRepo.get_refresh_token_ids(
        user_id=user_info.id,
        db_session=db_session
    )
    await put_tokens_in_black_list(refresh_token_ids)
    await SessionRepo.close_all(
        user_id=user_info.id,
        db_session=db_session
    )
    return BaseResponse(message='Все сессии успешно закрыты')


//src.routers.reports.views.py
@router.post(
    '/articles/{article_id}/report/',
    response_model=DataResponse.single_by_key(
        'report',
        ReportOutScheme
    ),
    responses=get_responses(400, 401, 403, 409)
)
async def create_report(
        report_data: CreateReportScheme,
        report: Report | None = Depends(get_report(owner_only=True)),
        article_id: uuid.UUID = Path(),
        db_session: AsyncSession = Depends(get_session),
        user_info: UserInfo = Depends(JWTCookie(roles=[Role.user])),
):
    article = await ArticleRepo.get_by_id(
        article_id=article_id,
        db_session=db_session
    )
    if article.original_article_id is None:
        raise HTTPException(
            status_code=status.HTTP_400_BAD_REQUEST,
            detail='Жаловаться можно только на переводы'
        )
    report = await ReportRepo.create(
        article_id=article_id,
        report_data=report_data,
        db_session=db_session
    )
    return DataResponse(
        data={
            'report': ReportOutScheme.create(report)
        }
    )




@router.put(
    '/articles/{article_id}/report/',
    response_model=DataResponse.single_by_key(
        'report',
        ReportOutScheme
    ),
    responses=get_responses(400, 401, 403, 404)
)
async def update_report(
        report_data: EditReportScheme,
        report: Report | None = Depends(get_report(owner_only=True)),
        db_session: AsyncSession = Depends(get_session),
        user_info: UserInfo = Depends(JWTCookie(roles=[Role.user])),
):
    if not report:
        raise report_not_found_error
    report = await ReportRepo.update(
        report=report,
        report_data=report_data,
        db_session=db_session
    )
    return DataResponse(
        data={
            'report': ReportOutScheme.create(report)
        }
    )




@router.patch(
    '/articles/{article_id}/report/status/',
    response_model=DataResponse.single_by_key(
        'report',
        ReportOutScheme
    ),
    responses=get_responses(400, 401, 403, 404)
)
async def update_report_status(
        new_status: ReportStatus,
        article_id: uuid.UUID = Path(),
        report: Report | None = Depends(get_report(owner_only=False)),
        db_session: AsyncSession = Depends(get_session),
        user_info: UserInfo = Depends(JWTCookie(roles=[
            Role.user, Role.moderator
        ])),
):
    if not report:
        raise report_not_found_error
    if report.status != ReportStatus.open:
        raise HTTPException(
            status_code=status.HTTP_400_BAD_REQUEST,
            detail='Жалоба уже закрыта'
        )
    if (
        user_info.role == Role.user and new_status != ReportStatus.closed or
        user_info.role == Role.moderator and new_status not
        in [ReportStatus.rejected, ReportStatus.satisfied]
    ):
        raise HTTPException(
            status_code=status.HTTP_403_FORBIDDEN,
            detail='Действие запрещено'
        )
    return DataResponse(
        data={
            'report': ReportOutScheme.create(
                await ReportRepo.update_status(
                    report=report,
                    new_status=new_status,
                    user_id=user_info.id,
                    db_session=db_session
                )
            )
        }
    )


@router.get(
    '/articles/{article_id}/report/comments/',
    response_model=SimpleListResponse[CommentOutScheme],
    responses=get_responses(400, 401, 403, 409)
)
async def get_comments(
        report: Report | None = Depends(get_report(owner_only=False)),
        user_info: UserInfo = Depends(JWTCookie(roles=[
            Role.user, Role.moderator
        ])),
        db_session: AsyncSession = Depends(get_session)
):
    if not report:
        raise report_not_found_error
    return SimpleListResponse[CommentOutScheme].from_list(
        await ReportRepo.get_comments(
            article_id=report.article_id,
            db_session=db_session
        )
    )


@router.post(
    '/articles/{article_id}/report/comments/',
    response_model=DataResponse.single_by_key(
        'comment',
        CommentOutScheme
    ),
    responses=get_responses(400, 401, 403, 404)
)
async def create_comment(
        comment_data: CreateCommentScheme,
        report: Report | None = Depends(get_report(owner_only=False)),
        user_info: UserInfo = Depends(JWTCookie(roles=[
            Role.user, Role.moderator
        ])),
        db_session: AsyncSession = Depends(get_session)
):
    if not report or report.status != ReportStatus.open:
        raise report_not_found_error
    comment = await ReportRepo.create_comment(
        report_id=report.id,
        sender_id=user_info.id,
        text=comment_data.text,
        db_session=db_session
    )
    await db_session.refresh(report)
    redis_client = RedisHandler().client
    comment_scheme = CommentOutScheme(
            text=comment.text,
            sender_id=str(comment.sender_id),
            sender_name=(await UserRepo.get_by_id(
                user_id=user_info.id,
                db_session=db_session
            )).name,
            created_at=comment.created_at
        )
    await redis_client.publish(
        f'comments_{str(report.article_id)}',
        comment_scheme.model_dump_json()
    )
    return DataResponse(
        data={
            'comment': comment_scheme
        }
    )


//src.routers.translation.views.py
@router.post(
    '/',
    response_model=BaseResponse,
    responses=get_responses(400, 401, 403, 404)
)
async def create_translation(
        translation_data: CreateTranslationScheme,
        db_session: AsyncSession = Depends(get_session),
        user_info: UserInfo = Depends(JWTCookie(roles=[Role.user]))
):
    article = await ArticleRepo.get_by_id(
        article_id=translation_data.article_id,
        db_session=db_session
    )
    if not article or article.user_id != user_info.id:
        raise HTTPException(
            status_code=status.HTTP_404_NOT_FOUND,
            detail='Статья не найдена'
        )
    if article.original_article_id:
        raise HTTPException(
            status_code=status.HTTP_400_BAD_REQUEST,
            detail='Нельзя переводить перевод'
        )
    if not await ModelRepo.exists_by_id(
        model_id=translation_data.model_id,
        db_session=db_session
    ):
        raise HTTPException(
            status_code=status.HTTP_404_NOT_FOUND,
            detail='Модель не существует'
        )
    if not await PromptRepo.exists_by_id(
        prompt_id=translation_data.prompt_id,
        db_session=db_session
    ):
        raise HTTPException(
            status_code=status.HTTP_404_NOT_FOUND,
            detail='Промпт не существует'
        )


    producer = KafkaProducer(
        bootstrap_servers=KafkaConfig.address,
        topic=KafkaConfig.translation_topic
    )


    for target_language_id in translation_data.target_language_ids:
        if not await LanguageRepo.exists(
                language_id=target_language_id,
                db_session=db_session
        ):
            raise HTTPException(
                status_code=status.HTTP_404_NOT_FOUND,
                detail='Конечный язык не поддерживается'
            )
        task = await TaskRepo.create(
            task_data=CreateTaskScheme(
                article_id=translation_data.article_id,
                model_id=translation_data.model_id,
                prompt_id=translation_data.prompt_id,
                target_language_id=target_language_id
            ),
            db_session=db_session
        )
        message = TranslationMessage(task_id=task.id)
        await producer.send_message(
            message.model_dump(mode='json')
        )
        return BaseResponse(message='Перевод запущен. Ожидайте')




//src.routers.users.views.py
@router.patch(
    '/{user_id}/name/',
    response_model=BaseResponse,
    responses=get_responses(400, 401, 409)
)
async def change_name(
        request_data: UserUpdateNameScheme,
        user_id: uuid.UUID = Path(),
        user_info: UserInfo = Depends(JWTCookie()),
        db_session: AsyncSession = Depends(get_session)
):
    user = await UserRepo.get_by_id(
        user_id=user_info.id,
        db_session=db_session
    )
    if not user or user_id != user_info.id:
        raise HTTPException(
            status_code=status.HTTP_401_UNAUTHORIZED,
            detail='Пользователь не найден'
        )
    if user.name == request_data.name:
        raise HTTPException(
            status_code=status.HTTP_409_CONFLICT,
            detail='Новое имя не должно совпадать со старым'
        )
    user.name = request_data.name
    db_session.add(user)
    await db_session.commit()
    return BaseResponse(message='Имя успешно изменено')


@router.post(
    '/',
    response_model=DataResponse.single_by_key(
        'user',
        UserOutScheme
    ),
    responses=get_responses(400, 401, 403, 409)
)
async def create_user(
        new_user_data: CreateUserScheme,
        user_info: UserInfo = Depends(JWTCookie(roles=[Role.admin])),
        db_session: AsyncSession = Depends(get_session)
):
    user = await UserRepo.create(
        user_data=new_user_data,
        db_session=db_session
    )
    return DataResponse(
        data={
            'user': UserOutAdminScheme.model_validate(user)
        }
    )


@router.put(
    '/{user_id}/',
    response_model=DataResponse.single_by_key(
        'user',
        UserOutScheme
    ),
    responses=get_responses(400, 401, 403, 409)
)
async def update_user(
        new_user_info: EditUserScheme,
        user: User = Depends(get_user),
        user_info: UserInfo = Depends(JWTCookie(roles=[Role.admin])),
        db_session: AsyncSession = Depends(get_session),
):
    user = await UserRepo.update(
        user=user,
        new_data=new_user_info,
        db_session=db_session
    )
    return DataResponse(
        data={
            'user': UserOutAdminScheme.model_validate(user)
        }
    )


@router.delete(
    '/{user_id}/',
    responses=get_responses(400, 401, 403, 409)
)
async def delete_user(
        user: User = Depends(get_user),
        user_info: UserInfo = Depends(JWTCookie(roles=[Role.admin])),
        db_session: AsyncSession = Depends(get_session),
):
    await UserRepo.soft_delete(
        user=user,
        db_session=db_session
    )
    return BaseResponse(message='Пользователь удалён')




//src.database.models.py
class User(Base):
    __tablename__ = f'{Database.prefix}users'
    id: Mapped[uuid.UUID] = mapped_column(
        UUID(as_uuid=True),
        primary_key=True,
        default=uuid.uuid4
    )
    name: Mapped[str] = mapped_column(
        String(20)
    )
    email: Mapped[str] = mapped_column(
        String,
        unique=True
    )
    email_verified: Mapped[bool] = mapped_column(
        Boolean,
        default=False
    )
    password_hash: Mapped[str] = mapped_column(
        String(60)
    )
    role: Mapped[Role] = mapped_column(
        Enum(Role, name='user_role'),
        default=Role.user
    )
    logged_with_provider: Mapped[str | None] = mapped_column(
        String,
        nullable=True,
        comment='External OAuth provider name user has registered with'
    )
    provider_id: Mapped[str | None] = mapped_column(
        String,
        nullable=True,
        comment='User\'s ID from OAuth provider user has registered with'
    )
    created_at: Mapped[datetime.datetime] = mapped_column(
        DateTime,
        default=get_utc_now
    )
    deleted_at: Mapped[datetime.datetime | None] = mapped_column(
        DateTime,
        nullable=True
    )




class Session(Base):
    __tablename__ = f'{Database.prefix}sessions'
    id: Mapped[uuid.UUID] = mapped_column(
        UUID(as_uuid=True),
        primary_key=True,
        default=uuid.uuid4
    )
    user_id: Mapped[uuid.UUID] = mapped_column(
        ForeignKey(f'{Database.prefix}users.id', ondelete='CASCADE')
    )
    ip: Mapped[str] = mapped_column(
        String(15)
    )
    user_agent: Mapped[str] = mapped_column(
        String(100)
    )
    is_closed: Mapped[bool] = mapped_column(
        Boolean,
        default=False
    )
    refresh_token_id: Mapped[uuid.UUID] = mapped_column(
        UUID(as_uuid=True),
    )
    created_at: Mapped[datetime.datetime] = mapped_column(
        DateTime,
        default=get_utc_now
    )
    closed_at: Mapped[datetime.datetime | None] = mapped_column(
        DateTime,
        nullable=True
    )




class ConfirmationCode(Base):
    __tablename__ = f'{Database.prefix}confirmation_codes'
    id: Mapped[int] = mapped_column(
        Integer,
        primary_key=True
    )
    code: Mapped[str] = mapped_column(
        String,
        unique=True,
        comment='The value of the code'
    )
    reason: Mapped[ConfirmationType] = mapped_column(
        Enum(ConfirmationType),
        default=ConfirmationType.registration
    )
    user_id: Mapped[uuid.UUID] = mapped_column(
        ForeignKey(f'{User.__tablename__}.id', ondelete='CASCADE')
    )
    expired_at: Mapped[datetime.datetime] = mapped_column(
        DateTime
    )
    is_used: Mapped[bool] = mapped_column(
        Boolean,
        default=False
    )
    created_at: Mapped[datetime.datetime] = mapped_column(
        DateTime,
        default=get_utc_now
    )




class Language(Base):
    __tablename__ = f'{Database.prefix}languages'
    id: Mapped[int] = mapped_column(
        Integer,
        primary_key=True
    )
    name: Mapped[str] = mapped_column(
        String,
        unique=True
    )
    iso_code: Mapped[str] = mapped_column(
        String,
        unique=True
    )




class Article(Base):
    __tablename__ = f'{Database.prefix}articles'
    id: Mapped[uuid.UUID] = mapped_column(
        UUID(as_uuid=True),
        primary_key=True,
        default=uuid.uuid4
    )
    title: Mapped[str] = mapped_column(String(50))
    text: Mapped[str] = mapped_column(Text)
    user_id: Mapped[uuid.UUID] = mapped_column(
        ForeignKey(f'{User.__tablename__}.id', ondelete='CASCADE')
    )
    language_id: Mapped[int | None] = mapped_column(
        ForeignKey(f'{Language.__tablename__}.id', ondelete='CASCADE'),
        nullable=True
    )
    original_article_id: Mapped[uuid.UUID | None] = mapped_column(
        ForeignKey(f'{Database.prefix}articles.id', ondelete='CASCADE'),
        nullable=True
    )
    like: Mapped[bool | None] = mapped_column(
        Boolean,
        nullable=True
    )
    created_at: Mapped[datetime.datetime] = mapped_column(
        DateTime,
        default=get_utc_now
    )
    deleted_at: Mapped[datetime.datetime | None] = mapped_column(
        DateTime,
        nullable=True
    )


    report: Mapped['Report'] = relationship(
        'Report',
        back_populates='article',
        cascade='all, delete-orphan',
        uselist=False,
        lazy='joined'
    )
    language: Mapped[Language] = relationship(
        'Language',
        uselist=False,
        lazy='joined'
    )
    original_article: Mapped['Article'] = relationship(
        'Article',
        uselist=False,
        lazy='joined'
    )




class ReportReason(Base):
    __tablename__ = f'{Database.prefix}report_reasons'
    id: Mapped[int] = mapped_column(
        Integer,
        primary_key=True
    )
    text: Mapped[str] = mapped_column(
        String,
        unique=True
    )
    order_position: Mapped[int] = mapped_column(
        Integer,
        unique=True
    )




class Report(Base):
    __tablename__ = f'{Database.prefix}reports'
    id: Mapped[uuid.UUID] = mapped_column(
        UUID(as_uuid=True),
        primary_key=True,
        default=uuid.uuid4
    )
    text: Mapped[str] = mapped_column(
        String(1024)
    )
    article_id: Mapped[uuid.UUID] = mapped_column(
        ForeignKey(f'{Article.__tablename__}.id', ondelete='CASCADE')
    )
    status: Mapped[ReportStatus] = mapped_column(
        Enum(ReportStatus),
        default=ReportStatus.open
    )
    closed_by_user_id: Mapped[uuid.UUID | None] = mapped_column(
        ForeignKey(f'{User.__tablename__}.id', ondelete='CASCADE'),
        nullable=True
    )
    reason_id: Mapped[int] = mapped_column(
        ForeignKey(f'{ReportReason.__tablename__}.id', ondelete='CASCADE')
    )
    created_at: Mapped[datetime.datetime] = mapped_column(
        DateTime,
        default=get_utc_now
    )
    closed_at: Mapped[datetime.datetime | None] = mapped_column(
        DateTime,
        nullable=True
    )


    article: Mapped[Article] = relationship(
        'Article',
        back_populates='report',
        uselist=False,
        lazy='joined',
    )
    closed_by_user: Mapped[User] = relationship(
        'User',
        uselist=False,
        lazy='joined',
    )
    reason: Mapped[ReportReason] = relationship(
        'ReportReason',
        uselist=False,
        lazy='joined',
    )




class Comment(Base):
    __tablename__ = f'{Database.prefix}report_comments'
    id: Mapped[uuid.UUID] = mapped_column(
        UUID(as_uuid=True),
        primary_key=True,
        default=uuid.uuid4
    )
    text: Mapped[str] = mapped_column(
        String(100)
    )
    sender_id: Mapped[uuid.UUID] = mapped_column(
        ForeignKey(f'{User.__tablename__}.id', ondelete='CASCADE')
    )
    report_id: Mapped[uuid.UUID] = mapped_column(
        ForeignKey(f'{Report.__tablename__}.id', ondelete='CASCADE')
    )
    created_at: Mapped[datetime.datetime] = mapped_column(
        DateTime,
        default=get_utc_now
    )




class StylePrompt(Base):
    __tablename__ = f'{Database.prefix}style_prompts'
    id: Mapped[int] = mapped_column(
        Integer,
        primary_key=True
    )
    title: Mapped[str] = mapped_column(
        String(20),
        unique=True
    )
    text: Mapped[str] = mapped_column(
        String,
        unique=True
    )
    created_at: Mapped[datetime.datetime] = mapped_column(
        DateTime,
        default=get_utc_now
    )
    deleted_at: Mapped[datetime.datetime | None] = mapped_column(
        DateTime,
        nullable=True
    )




class AIModel(Base):
    __tablename__ = f'{Database.prefix}ai_models'
    id: Mapped[int] = mapped_column(
        Integer,
        primary_key=True
    )
    show_name: Mapped[str] = mapped_column(String(50), nullable=False)
    name: Mapped[str] = mapped_column(String, nullable=False)
    provider: Mapped[str] = mapped_column(String, nullable=False)
    created_at: Mapped[datetime.datetime] = mapped_column(
        DateTime,
        default=get_utc_now,
        nullable=False
    )
    deleted_at: Mapped[datetime.datetime | None] = mapped_column(
        DateTime,
        nullable=True
    )




class TranslationConfig(Base):
    __tablename__ = f'{Database.prefix}configs'
    id: Mapped[int] = mapped_column(
        Integer,
        primary_key=True
    )
    user_id: Mapped[uuid.UUID] = mapped_column(
        ForeignKey(f'{User.__tablename__}.id', ondelete='CASCADE')
    )
    prompt_id: Mapped[int | None] = mapped_column(
        ForeignKey(
            f'{StylePrompt.__tablename__}.id',
            ondelete='CASCADE'
        ),
        nullable=True
    )
    name: Mapped[str] = mapped_column(
        String(20),
    )
    language_ids: Mapped[list[int]] = mapped_column(ARRAY(Integer))
    model_id: Mapped[int | None] = mapped_column(
        ForeignKey(f'{AIModel.__tablename__}.id', ondelete='CASCADE'),
        nullable=True
    )
    created_at: Mapped[datetime.datetime] = mapped_column(
        DateTime,
        default=get_utc_now
    )
    deleted_at: Mapped[datetime.datetime | None] = mapped_column(
        DateTime,
        nullable=True
    )




class TranslationTask(Base):
    __tablename__ = f'{Database.prefix}translation_tasks'
    id: Mapped[uuid.UUID] = mapped_column(
        UUID(as_uuid=True),
        primary_key=True,
        default=uuid.uuid4
    )
    article_id: Mapped[uuid.UUID] = mapped_column(
        ForeignKey(f'{Article.__tablename__}.id', ondelete='CASCADE')
    )
    target_language_id: Mapped[int] = mapped_column(
        ForeignKey(f'{Language.__tablename__}.id', ondelete='CASCADE')
    )
    prompt_id: Mapped[int] = mapped_column(
        ForeignKey(
            f'{StylePrompt.__tablename__}.id', ondelete='CASCADE'
        )
    )
    model_id: Mapped[int] = mapped_column(
        ForeignKey(
            f'{AIModel.__tablename__}.id', ondelete='CASCADE'
        )
    )
    status: Mapped[TranslationTaskStatus] = mapped_column(
        Enum(TranslationTaskStatus),
        default=TranslationTaskStatus.created
    )
    data: Mapped[dict] = mapped_column(
        JSONB,
        nullable=True,
        comment='Additional data related to the translation task '
                '(e.g., errors or metadata)'
    )
    translated_article_id: Mapped[uuid.UUID | None] = mapped_column(
        ForeignKey(f'{Article.__tablename__}.id', ondelete='CASCADE'),
        nullable=True
    )
    created_at: Mapped[datetime.datetime] = mapped_column(
        DateTime,
        default=get_utc_now
    )
    deleted_at: Mapped[datetime.datetime | None] = mapped_column(
        DateTime,
        nullable=True
    )




class Notification(Base):
    __tablename__ = f'{Database.prefix}notifications'
    id: Mapped[uuid.UUID] = mapped_column(
        UUID(as_uuid=True),
        primary_key=True,
        default=uuid.uuid4
    )
    title: Mapped[str] = mapped_column(
        String
    )
    text: Mapped[str] = mapped_column(
        String
    )
    user_id: Mapped[uuid.UUID] = mapped_column(
        ForeignKey(f'{User.__tablename__}.id', ondelete='CASCADE')
    )
    type: Mapped[NotificationType] = mapped_column(
        Enum(NotificationType)
    )
    read_at: Mapped[datetime.datetime | None] = mapped_column(
        DateTime,
        nullable=True
    )
    created_at: Mapped[datetime.datetime] = mapped_column(
        DateTime,
        default=get_utc_now
    )




//tests.hot_load.py
import asyncio
import datetime
import logging
import multiprocessing
import random
import statistics
import time
from collections import defaultdict
from functools import wraps


import httpx
logger = logging.getLogger(__name__)


class HotLoad:
    def __init__(
            self,
            duration: datetime.timedelta,
            processes_number: int = 1,
            workers_number: int = 1,
    ):
        self.duration = duration
        self.deadline = datetime.datetime.now() + duration
        self.processes_number = processes_number
        self.workers_number = workers_number
        self.headers = {}
        self.tasks = []
        self.errors = 0  # find usage
        self.on_startup_callable = None
        self.on_teardown_callable = None
    def task(self, func):
        @wraps(func)
        async def wrapper(*args, **kwargs):
            return await func(*args, **kwargs)
        self.tasks.append(wrapper)
        return wrapper
    def on_startup(self, func):
        @wraps(func)
        async def wrapper(*args, **kwargs):
            return await func(*args, **kwargs)
        self.on_startup_callable = wrapper
        return wrapper
    def on_teardown(self, func):
        @wraps(func)
        async def wrapper(*args, **kwargs):
            return await func(*args, **kwargs)
        self.on_teardown_callable = wrapper
        return wrapper
    @staticmethod
    def get_median(results) -> float:
        return statistics.median(results) if results else 0
    @staticmethod
    def get_timestamp_now():
        return datetime.datetime.now().strftime('%Y-%m-%d %H:%M:%S')
    async def run_worker(
            self,
            worker_id: int
    ) -> int:
        logger.info('Running worker %s', worker_id)
        stats = defaultdict(list)
        total_requests = 0
        async with httpx.AsyncClient(headers=self.headers) as client:
            while self.deadline > datetime.datetime.now():
                try:
                    start = time.time()
                    await self.tasks[
                        random.randint(0, len(self.tasks) - 1)
                    ](
                        client,
                        worker_id
                    )
                    end = time.time()
                    delta = end - start
                    current_timestamp = self.get_timestamp_now()
                    stats[current_timestamp].append(delta)
                except Exception as e:
                    logger.exception(e)
                    self.errors += 1
                total_requests += 1
        return total_requests
    async def run_workers(self, worker_start_id: int) -> int:
        results = await asyncio.gather(*[
            self.run_worker(worker_id=i + worker_start_id)
            for i in range(self.workers_number)
        ])
        return sum(results)
    def run_process(self, process_number: int, *args) -> int:
        worker_start_id = process_number * self.workers_number
        loop = asyncio.get_event_loop()
        result = loop.run_until_complete(self.run_workers(worker_start_id))
        return result


    async def run(self) -> float:
        if self.on_startup_callable:
            self.headers = await self.on_startup_callable()


        with multiprocessing.Pool(processes=self.processes_number) as pool:
            results = pool.map(self.run_process, range(self.processes_number))
        mean_rps = sum(results) / self.duration.total_seconds()
        if self.on_teardown_callable:
            await self.on_teardown_callable()
        return mean_rps
\end{lstlisting}