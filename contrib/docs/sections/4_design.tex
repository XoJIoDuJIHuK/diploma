\clearpage
\SectionWithSubsection{Проектирование веб-приложения}
\SubsectionWithParagraph{Функциональность веб-приложения}

Функциональные возможности веб-приложения представлены в диаграмме вариантов использования, представленной на рисунке \ref{img:usecase}.

\begin{figure}[H]
    \centering
    \borderedimage[0.95\linewidth]{img/use-case.png}{yes}
    \caption{Диаграмма вариантов использования веб-приложения \label{img:usecase}}
\end{figure}

Перечень ролей и их назначение приведены в таблице \ref{tab:user_roles}.

\begin{longtable}{|p{82mm}|p{83mm}|}
    \caption[]{Назначение ролей пользователей в веб-приложении \label{tab:user_roles}} \\ \hline
    \endfirsthead
    \multicolumn{2}{l}{Продолжение таблицы \thetable} \endhead
    Роль & Назначение \\ \hline
    Гость & Регистрация и аутентификация \\ \hline
    Пользователь & Загрузка и запуск перевода статей, получение переводов, создание жалоб на переводы своих статей \\ \hline
    Модератор & Рассмотрение жалоб на переводы \\ \hline
    Администратор & Управление пользователями, запросами перевода, моделями перевода \\ \hline
\end{longtable}

Функциональные возможности пользователя с ролью «Гость» представлены в таблице \ref{tab:guest_functions}.

\begin{longtable}{|p{82mm}|p{83mm}|}
    \caption[]{Функциональные возможности пользователя с ролью «Гость» \label{tab:guest_functions}} \\ \hline
    \endfirsthead
    \multicolumn{2}{l}{Продолжение таблицы \thetable} \endhead
    Вариант использования & Пояснение \\ \hline
    12 Регистрироваться & Гость может создать учётную запись при помощи электронной почты и пароля или OAuth 2.0-провайдера \\ \hline
    13 Аутентифицироваться & Гость может аутентифицироваться при помощи электронной почты и пароля или OAuth 2.0-провайдера \\ \hline
\end{longtable}

После аутентификации гость становится либо пользователем, либо модератором, либо администратором. По этой причине пользователю недоступна аутентификация. Функциональные возможности пользователя с ролью «Пользователь» представлены в таблице \ref{tab:user_functions}.

\begin{longtable}{|p{83mm}|p{83mm}|}
    \caption[]{Функциональные возможности пользователя с ролью «Пользователь» \label{tab:user_functions}} \\ \hline
    \endfirsthead
    \multicolumn{2}{l}{Продолжение таблицы \thetable} \endhead
    Вариант использования & Пояснение \\ \hline
    1 Изменять учётную запись & Изменять своё имя и пароль \\ \hline
    2 Просматривать список открытых сессий & Получать список открытых сессий \\ \hline
    3 Завершать открытые сессии & Блокировать доступ для всех открытых сессий \\ \hline
    4 Изменять список исходных статей & Загружать из файла или вводить с клавиатуры исходные статьи, получать список исходных статей, изменять содержимое исходных статей, удалять их \\ \hline
    5 Изменять список переведённых статей & Запускать перевод исходных статей, получать их список, оставлять оценку переводам статей, удалять переводы статей \\ \hline
    6 Изменять список жалоб на переводы своих статей & Создавать жалобы на переводы своих статей, получать их список, закрывать открытые жалобы на переводы своих статей \\ \hline
    7 Просматривать свои уведомления & Получать список непрочитанных уведомлений \\ \hline
    8 Изменять список комментариев к жалобам на переводы своих статей & Получать список комментариев, создавать комментарии к открытым жалобам на переводы своих статей \\ \hline
    9 Получить список комментариев к жалобе & Получить список комментариев к одной из своих жалоб \\ \hline
    10 Создать комментарий & Создать комментарий к одной из своих жалоб \\ \hline
    11 Изменять список настроек переводчика & Получать список своих конфигураций, создавать новые, обновлять и удалять существующие конфигурации \\ \hline
    23 Покупать токены & Совершать покупки токенов через внешнюю систему оплаты для последующего использования в переводах \\ \hline
\end{longtable}

Модератор может рассматривать жалобы пользователей. Его функциональные возможности представлен в таблице \ref{tab:moderator_functions}.

\begin{longtable}{|p{82mm}|p{83mm}|}
    \caption[]{Функциональные возможности пользователя с ролью «Модератор» \label{tab:moderator_functions}} \\ \hline
    \endfirsthead
    \multicolumn{2}{l}{Продолжение таблицы \thetable} \endhead
    Вариант использования & Пояснение \\ \hline
    1 Изменять учётную запись & Изменять своё имя и пароль \\ \hline
    2 Просматривать список открытых сессий & Получать список открытых сессий \\ \hline
    3 Завершать открытые сессии & Блокировать доступ для всех открытых сессий \\ \hline
    14 Изменять список открытых жалоб & Получать список открытых жалоб на переводы, получать списки комментариев и создавать новые комментарии к ним, принимать или отклонять жалобы \\ \hline
    15 Создавать комментарии для жалоб & Создавать комментарии для открытой жалобы \\ \hline
\end{longtable}

Функциональные возможности администратора представлен в таблице \ref{tab:admin_functions}.

\begin{longtable}{|p{82mm}|p{83mm}|}
    \caption[]{Функциональные возможности пользователя с ролью «Администратор» \label{tab:admin_functions}} \\ \hline
    \endfirsthead
    \multicolumn{2}{l}{Продолжение таблицы \thetable} \endhead
    Вариант использования & Пояснение \\ \hline
    1 Изменять учётную запись & Изменять своё имя и пароль \\ \hline
    2 Просматривать список открытых сессий & Получать список открытых сессий \\ \hline
    3 Завершать открытые сессии & Блокировать доступ для всех открытых сессий \\ \hline
    16 Просматривать статистику жалоб & Получать данные о том, какая часть переводов при помощи каждой модели получает жалобы и какая их доля удовлетворяется модераторами \\ \hline
    17 Изменять список стилей перевода & Создавать новые стили, обновлять и удалять существующие \\ \hline
    18 Изменять список моделей перевода & Добавлять информацию о новых моделях, изменять и удалять существующие записи \\ \hline
    19 Изменять список пользователей & Получать список пользователей, создавать новых, изменять и удалять существующих \\ \hline
    20 Создавать пользователей & Создавать объекты пользователей с ролью пользователя \\ \hline
    21 Создавать модераторов & Создавать объекты пользователей с ролью модератора \\ \hline
    22 Создавать администраторов & Создавать объекты пользователей с ролью администратора \\ \hline
\end{longtable}

Таким образом, пользователю доступны базовые операции, такие как операции над статьями и настройками перевода, модераторы могут управлять жалобами, а администраторы – управлять пользователями, моделями, запросами перевода и просматривать статистику жалоб на переводы.

\SubsectionBetweenParagraphs{Структура базы данных}

Согласно схеме вариантов использования была создана база данных. Её структура представлена на рисунке \ref{img:db_logic}.

\begin{figure}[H]
    \centering
    \borderedimage[1\linewidth]{img/db-logic.png}{yes}
    \caption{Логическая схема базы данных \label{img:db_logic}}
\end{figure}

База данных содержит тринадцать таблиц, хранящих информацию о пользователях, сессиях, статьях и прочих данных. Типы данных были выбраны согласно [1]. Назначение таблиц базы данных представлено в таблице \ref{tab:tables_purpose}.

\begin{longtable}{|p{82mm}|p{83mm}|}
    \caption[]{Назначение таблиц базы данных \label{tab:tables_purpose}} \\ \hline
    \endfirsthead
    \multicolumn{2}{l}{Продолжение таблицы \thetable} \endhead
    Таблица & Назначение \\ \hline
    Users & Хранит информацию о пользователях (имя, адрес электронной почты и хеш пароля для аутентификации и так далее) \\ \hline
    Sessions & Хранит информацию о сессиях пользователей (идентификатор пользователя, флаг активности, время создания и так далее) \\ \hline
    Confirmation\_codes & Хранит информацию о кодах подтверждения адреса электронной почты и сброса пароля \\ \hline
    Languages & Хранит информацию о доступных для перевода языках (название, ISO код) \\ \hline
    Articles & Хранит информацию о статьях (заголовок, текст, идентификатор пользователя и так далее) \\ \hline
    Report\_reasons & Хранит информацию о доступных причинах для жалобы на перевод статьи (текст, позиция в списке для сортировки) \\ \hline
    Reports & Хранит информацию о жалобах на переводы статей (идентификатор статьи, текст, идентификатор, причина и так далее) \\ \hline
    Report\_comments & Хранит информацию о комментариях к жалобам на переводы статей (текст, идентификатор пользователя, идентификатор жалобы, дата и время создания) \\ \hline
    Style\_prompts & Хранит информацию о запросах перевода с разными стилями (название, текст и так далее) \\ \hline
    AI\_Models & Хранит информацию о моделях искусственного интеллекта, использующихся для перевода (название, поставщик и так далее) \\ \hline
    Configs & Хранит информацию о конфигурациях переводчика, которые могут использоваться пользователями для упрощения запуска перевода своих статей (идентификаторы запроса перевода, модели, языков и так далее) \\ \hline
    Translation\_tasks & Хранит информацию о задачах перевода, которые считываются отдельным процессом и выполняются им (идентификаторы статьи, модели, исходного и конечного языков, статус и так далее) \\ \hline
    Notifications & Хранит информацию об уведомлениях пользователей (идентификатор пользователя, текст, тип уведомления и так далее) \\ \hline
\end{longtable}

Описание столбцов таблицы Users представлено в таблице \ref{tab:structure_users}.

\begin{longtable}{|p{55mm}|p{55mm}|p{55mm}|}
    \caption[]{Описание таблицы Users \label{tab:structure_users}} \\ \hline
    \endfirsthead
    \multicolumn{3}{l}{Продолжение таблицы \thetable} \endhead
    Название столбца & Тип данных & Описание \\ \hline
    id & uuid & Идентификатор пользователя, первичный ключ  \\ \hline
    name & varchar (20) & Имя пользователя \\ \hline
    email & varchar (50) & Адрес электронной почты пользователя \\ \hline
    email\_verified & boolean & Флаг, указывающий, был ли подтверждён адрес электронной почты пользователя \\ \hline
    password\_hash & varchar (60) & Хеш пароля соискателя \\ \hline
    role & enum user\_role & Роль пользователя (пользователь, модератор, администратор) \\ \hline
    logged\_with\_provider & varchar & Название провайдера OAuth 2.0, использовавшегося для регистрации \\ \hline
    provider\_id & varchar & Идентификатор пользователя, полученный от провайдера OAuth при регистрации \\ \hline
    created\_at & timestamp without timezone & Дата и время создания пользователя без часового пояса \\ \hline
    deleted\_at & timestamp without timezone & Дата и время удаления пользователя без часового пояса \\ \hline
\end{longtable}

Таблица Sessions хранит данные о сессиях пользователей. Описание её столбцов представлено в таблице \ref{tab:structure_sessions}.

\begin{longtable}{|p{55mm}|p{55mm}|p{55mm}|}
    \caption[]{Описание таблицы Sessions \label{tab:structure_sessions}} \\ \hline
    \endfirsthead
    \multicolumn{3}{l}{Продолжение таблицы \thetable} \endhead
    Название столбца & Тип данных & Описание \\ \hline
    id & uuid & Идентификатор сессии, первичный ключ \\ \hline
    user\_id & uuid & Идентификатор пользователя, который создал данную сессию, внешний ключ \\ \hline
    ip & varchar (15) & IPv4 адрес узла, из которого была открыта сессия \\ \hline
    user\_agent & varchar (100) & User agent клиента (например, браузера) \\ \hline
    is\_closed & boolean & Флаг, указывающий, была ли сессия закрыта \\ \hline
    refresh\_token\_id & uuid & Идентификатор refresh токена, связанного с данной сессией \\ \hline
    created\_at & timestamp without timezone & Дата и время создания сессии без часового пояса \\ \hline
    closed\_at & timestamp without timezone & Дата и время закрытия сессии без часового пояса \\ \hline
\end{longtable}

Описание столбцов таблицы Confirmation\_codes представлено в таблице \ref{tab:structure_conf_codes}.

\begin{longtable}{|p{55mm}|p{55mm}|p{55mm}|}
    \caption[]{Назначение таблиц базы данных \label{tab:structure_conf_codes}} \\ \hline
    \endfirsthead
    \multicolumn{3}{l}{Продолжение таблицы \thetable} \endhead
    Название столбца & Тип данных & Описание \\ \hline
    id & integer & Идентификатор кода, первичный ключ \\ \hline
    code & varchar & Строковое значение кода \\ \hline
    reason & enum confirmationtype & Тип кода (подтверждение адреса электронной почты, сброс пароля) \\ \hline
    user\_id & uuid & Идентификатор пользователя, для которого предназначен данный код подтверждения, внешний ключ \\ \hline
    expired\_at & timestamp without timezone & Временная отметка, после которой код будет считаться истёкшим \\ \hline
    is\_used & boolean & Флаг, указывающий, был ли код использован \\ \hline
    created\_at & timestamp without timezone & Дата и время создания кода без часового пояса \\ \hline
\end{longtable}

Таблица Languages хранит информацию о языках, доступных для перевода. Описание её столбцов представлено в таблице \ref{tab:structure_languages}.

\begin{longtable}{|p{55mm}|p{55mm}|p{55mm}|}
    \caption[]{Назначение таблиц базы данных \label{tab:structure_languages}} \\ \hline
    \endfirsthead
    \multicolumn{3}{l}{Продолжение таблицы \thetable} \endhead
    Название столбца & Тип данных & Описание \\ \hline
    id & integer & Идентификатор языка, первичный ключ  \\ \hline
    name & varchar & Отображаемое название языка \\ \hline
    iso\_code & varchar & ISO код языка \\ \hline
\end{longtable}

Таблица Articles хранит информацию об исходных и переведённых статьях. Описание её столбцов представлено в таблице \ref{tab:structure_articles}.

\begin{longtable}{|p{55mm}|p{55mm}|p{55mm}|}
    \caption[]{Назначение таблиц базы данных \label{tab:structure_articles}} \\ \hline
    \endfirsthead
    \multicolumn{3}{l}{Продолжение таблицы \thetable} \endhead
    Название столбца & Тип данных & Описание \\ \hline
    id & uuid & Идентификатор статьи, первичный ключ \\ \hline
    title & varchar (50) & Название статьи \\ \hline
    text & text & Текст статьи \\ \hline
    user\_id & uuid & Идентификатор пользователя, которому принадлежит статья, внешний ключ \\ \hline
    language\_id & integer & Идентификатор языка статьи, внешний ключ \\ \hline
    original\_article\_id & uuid & Идентификатор статьи, переводом которой является данная статья, внешний ключ \\ \hline
    like & boolean & Флаг, указывающий, какую оценку пользователь поставил переводу (положительную, отрицательную, не поставил оценку) \\ \hline
    created\_at & timestamp without timezone & Дата и время создания статьи без часового пояса \\ \hline
    deleted\_at & timestamp without timezone & Дата и время удаления статьи без часового пояса \\ \hline
\end{longtable}

Описание столбцов таблицы Report\_reasons представлено в таблице \ref{tab:structure_report_reasons}.

\begin{longtable}{|p{55mm}|p{55mm}|p{55mm}|}
    \caption[]{Назначение таблиц базы данных \label{tab:structure_report_reasons}} \\ \hline
    \endfirsthead
    \multicolumn{3}{l}{Продолжение таблицы \thetable} \endhead
    Название столбца & Тип данных & Описание \\ \hline
    id & integer & Идентификатор причины, первичный ключ \\ \hline
    text & varchar & Текст причины \\ \hline
    order\_position & integer & Положение причины в списке при сортировке \\ \hline
\end{longtable}

Описание столбцов таблицы Reports представлено в таблице \ref{tab:structure_reports}.

\begin{longtable}{|p{55mm}|p{55mm}|p{55mm}|}
    \caption[]{Назначение таблиц базы данных \label{tab:structure_reports}} \\ \hline
    \endfirsthead
    \multicolumn{3}{l}{Продолжение таблицы \thetable} \endhead
    Название столбца & Тип данных & Описание \\ \hline
    id & uuid & Идентификатор жалобы, первичный ключ \\ \hline
    text & varchar (1024) & Текст жалобы \\ \hline
    article\_id & uuid & Идентификатор статьи, на которую была оставлена жалоба, внешний ключ \\ \hline
    status & enum reportstatus & Статус жалобы (открыта, закрыта пользователем, отклонена, удовлетворена) \\ \hline
    closed\_by\_user\_id & uuid & Идентификатор пользователя, закрывшего жалобу (пользователь, которому принадлежит статья или модератор), внешний ключ \\ \hline
    reason\_id & int & Идентификатор причины, по которой была оставлена жалоба, внешний ключ \\ \hline
    created\_at & timestamp without timezone & Дата и время создания жалобы без часового пояса \\ \hline
    closed\_at & timestamp without timezone & Дата и время закрытия жалобы без часового пояса \\ \hline
\end{longtable}

Описание столбцов таблицы Report\_comments представлено в таблице \ref{tab:structure_report_comments}.

\begin{longtable}{|p{55mm}|p{55mm}|p{55mm}|}
    \caption[]{Назначение таблиц базы данных \label{tab:structure_report_comments}} \\ \hline
    \endfirsthead
    \multicolumn{3}{l}{Продолжение таблицы \thetable} \endhead
    Название столбца & Тип данных & Описание \\ \hline
    id & uuid & Идентификатор комментария, первичный ключ \\ \hline
    text & varchar (100) & Текст комментария \\ \hline
    sender\_id & uuid & Идентификатор пользователя, оставившего комментарий, внешний ключ \\ \hline
    report\_id & uuid & Идентификатор жалобы, к которой был оставлен комментарий, внешний ключ \\ \hline
    created\_at & timestamp without timezone & Дата и время создания комментария без часового пояса \\ \hline
\end{longtable}

Описание столбцов таблицы Style\_prompts представлено в таблице \ref{tab:structure_prompts}.

\begin{longtable}{|p{55mm}|p{55mm}|p{55mm}|}
    \caption[]{Назначение таблиц базы данных \label{tab:structure_prompts}} \\ \hline
    \endfirsthead
    \multicolumn{3}{l}{Продолжение таблицы \thetable} \endhead
    Название столбца & Тип данных & Описание \\ \hline
    id & integer & Идентификатор запроса, первичный ключ \\ \hline
    title & varchar (20) & Название запроса \\ \hline
    text & varchar & Текст запроса \\ \hline
    created\_at & timestamp without timezone & Дата и время создания запроса без часового пояса \\ \hline
    deleted\_at & timestamp without timezone & Дата и время удаления запроса без часового пояса \\ \hline
\end{longtable}

Описание столбцов таблицы AI\_Models представлено в таблице \ref{tab:structure_models}.

\begin{longtable}{|p{55mm}|p{55mm}|p{55mm}|}
    \caption[]{Назначение таблиц базы данных \label{tab:structure_models}} \\ \hline
    \endfirsthead
    \multicolumn{3}{l}{Продолжение таблицы \thetable} \endhead
    Название столбца & Тип данных & Описание \\ \hline
    id & integer & Идентификатор модели, первичный ключ \\ \hline
    show\_name & varchar (50) & Отображаемое название модели \\ \hline
    name & varchar & Название модели \\ \hline
    provider & varchar & Поставщик модели \\ \hline
    created\_at & timestamp without timezone & Дата и время создания записи о модели без часового пояса \\ \hline
    deleted\_at & timestamp without timezone & Дата и время удаления записи о модели без часового пояса \\ \hline
\end{longtable}

Таблица Configs хранит информацию о конфигурациях переводчика. Описание её столбцов представлено в таблице \ref{tab:structure_configs}.

\begin{longtable}{|p{55mm}|p{55mm}|p{55mm}|}
    \caption[]{Назначение таблиц базы данных \label{tab:structure_configs}} \\ \hline
    \endfirsthead
    \multicolumn{3}{l}{Продолжение таблицы \thetable} \endhead
    Название столбца & Тип данных & Описание \\ \hline
    id & integer & Идентификатор конфигурации, первичный ключ \\ \hline
    name & varchar (20) & Название конфигурации \\ \hline
    user\_id & uuid & Идентификатор пользователя, создавшего конфигурацию, внешний ключ \\ \hline
    prompt\_id & integer & Идентификатор запроса перевода, внешний ключ \\ \hline
    language\_ids & integer [] & Идентификаторы языков перевода \\ \hline
    model\_id & integer & Идентификатор модели перевода, внешний ключ \\ \hline
    created\_at & timestamp without timezone & Дата и время создания конфигурации без часового пояса \\ \hline
    deleted\_at & timestamp without timezone & Дата и время удаления конфигурации без часового пояса \\ \hline
\end{longtable}

Таблица Translation\_tasks хранит информацию о задачах перевода. Данная информация используется для определения текста исходной статьи, конечного языка и так далее. Описание столбцов таблицы представлено в таблице \ref{tab:structure_translation_tasks}.

\begin{longtable}{|p{55mm}|p{55mm}|p{55mm}|}
    \caption[]{Назначение таблиц базы данных \label{tab:structure_translation_tasks}} \\ \hline
    \endfirsthead
    \multicolumn{3}{l}{Продолжение таблицы \thetable} \endhead
    Название столбца & Тип данных & Описание \\ \hline
    id & uuid & Идентификатор задачи, первичный ключ \\ \hline
    article\_id & uuid & Идентификатор исходной статьи, внешний ключ \\ \hline
    target\_language\_id & integer & Идентификатор конечного языка, внешний ключ \\ \hline
    prompt\_id & integer & Идентификатор запроса перевода, внешний ключ \\ \hline
    model\_id & integer & Идентификатор модели перевода, внешний ключ \\ \hline
    status & enum translationtaskstatus & Статус задачи (создана, в процессе выполнения, завершена успешно, завершена с ошибкой) \\ \hline
    data & jsonb & Дополнительная информация о задаче (текст ошибки) \\ \hline
    translated\_article\_id & uuid & Идентификатор переведённой статьи, внешний ключ \\ \hline
    created\_at & timestamp without timezone & Дата и время создания задачи без часового пояса \\ \hline
    deleted\_at & timestamp without timezone & Дата и время удаления задачи без часового пояса \\ \hline
\end{longtable}

Описание столбцов таблицы Notifications представлено в таблице \ref{tab:structure_notifications}.

\begin{longtable}{|p{55mm}|p{55mm}|p{55mm}|}
    \caption[]{Назначение таблиц базы данных \label{tab:structure_notifications}} \\ \hline
    \endfirsthead
    \multicolumn{3}{l}{Продолжение таблицы \thetable} \endhead
    Название столбца & Тип данных & Описание \\ \hline
    id & uuid & Идентификатор уведомления, первичный ключ \\ \hline
    title & varchar & Заголовок уведомления \\ \hline
    text & varchar & Текст уведомления \\ \hline
    user\_id & uuid & Идентификатор пользователя, которому предназначено уведомление, внешний ключ \\ \hline
    type & enum notificationtype & Тип уведомления (информационное, предупреждение, ошибка) \\ \hline
    created\_at & timestamp without timezone & Дата и время создания записи о модели без часового пояса \\ \hline
    read\_at & timestamp without timezone & Дата и время удаления записи о модели без часового пояса \\ \hline
\end{longtable}

Назначение связей приведено в таблице \ref{tab:foreign_keys}.

\begin{longtable}{|p{82mm}|p{83mm}|}
    \caption[]{Назначение таблиц базы данных \label{tab:foreign_keys}} \\ \hline
    \endfirsthead
    \multicolumn{2}{l}{Продолжение таблицы \thetable} \endhead
    Связь & Назначение \\ \hline
    Users.id -- Notifications.user\_id & Идентификатор пользователя, которому адресовано уведомление \\ \hline
    Users.id -- Confirmation\_codes. user\_id & Идентификатор пользователя, которому предназначен код подтверждения \\ \hline
    Users.id -- Sessions.user\_id & Идентификатор пользователя, который создал сессию \\ \hline
    Users.id -- Articles.user\_id & Идентификатор пользователя, который загрузил статью или запустил перевод исходной статьи \\ \hline
    Users.id -- Configs.user\_id & Идентификатор пользователя, которому принадлежит конфигурация переводчика \\ \hline
    Users.id -- Commens.sender\_id & Идентификатор пользователя, отправившего комментарий \\ \hline
    Users.id -- Reports. closed\_by\_user\_id & Идентификатор пользователя, закрывшего жалобу (создавшего её пользователя или любого модератора) \\ \hline
    Report\_reasons.id -- Reports.reason\_id & Идентификатор причины, по которой была создана жалоба на перевод статьи \\ \hline
    Articles.id -- Articles. original\_article\_id & Идентификатор исходной статьи, из которой был создан перевод \\ \hline
    Articles.id -- Translation\_tasks. article\_id & Идентификатор статьи, которую необходимо перевести \\ \hline
    Articles.id -- Translation\_tasks. translated\_article\_id & Идентификатор перевода статьи \\ \hline
    Articles.id -- Reports.article\_id & Идентификатор перевода, на который была создана жалоба \\ \hline
    Languages.id -- Articles.language\_id & Идентификатор языка статьи \\ \hline
    Languages.id -- Translation\_tasks. target\_language\_id & Идентификатор конечного языка, на который необходимо перевести статью \\ \hline
    Reports.id -- Comments.report\_id & Идентификатор жалобы, под которой был оставлен комментарий \\ \hline
    AI\_Models.id -- Translation\_tasks. model\_id & Идентификатор записи о модели искусственного интеллекта, которая используется для перевода статьи \\ \hline
    AI\_Models.id -- Configs.model\_id & Идентификатор записи о модели искусственного интеллекта \\ \hline
    Style\_prompts.id -- Translation\_tasks. prompt\_id & Идентификатор запроса перевода, который используется для перевода статьи \\ \hline
    Style\_prompts.id -- Configs.prompt\_id & Идентификатор запроса перевода \\ \hline
\end{longtable}

Таким образом, была спроектирована база данных для долговременного хранения информации веб-приложения.

\SubsectionBetweenParagraphs{Архитектура веб-приложения}

Для обеспечения вспомогательных функций веб-приложения (отправка почты, выполнение перевода, отправка уведомлений между компонентами системы и так далее) используются дополнительные компоненты.

Для запуска многоконтейнерных Docker-приложений используется инструмент Docker Compose. Он управляет набором контейнеров, в которых работают прочие компоненты веб-приложения.

Для хранения данных используется СУБД PostgreSQL 17.

Для обслуживания веб-приложение и предоставления доступа к скомпилированному пакету фронтэнд-приложения, созданному с использованием Vue.js, используется веб-сервер Nginx.

Для асинхронного обмена сообщениями между компонентами системы используется брокер сообщений RabbitMQ.

Для обработки сообщений, передаваемых через RabbitMQ, используются два процесса-подписчика. Они принимают сообщения из очереди и обрабатывают поступившие команды, такие как перевод статьи и отправка электронной почты для подтверждения регистрации или сброса пароля.

Для быстрого доступа к данным, которые часто используются, например, идентификаторам закрытых сессий, и для передачи уведомлений пользователю используется in-memory база данных Redis.

Архитектура веб-приложения представлена на рисунке \ref{img:arch}.

\begin{figure}[H]
    \centering
    \borderedimage[1\linewidth]{img/architecture.png}{yes}
    \caption{Архитектура веб-приложения \label{img:arch}}
\end{figure}

Пояснение назначения каждого элемента веб-приложения на архитектурной диаграмме представлено в таблице \ref{tab:arch_elements_purpose}.

\begin{longtable}{|p{82mm}|p{83mm}|}
    \caption[]{Назначение элементов архитектурной схемы веб-приложения \label{tab:arch_elements_purpose}} \\ \hline
    \endfirsthead
    \multicolumn{2}{l}{Продолжение таблицы \thetable} \endhead
    Элемент & Назначение \\\hline
    Web Server (nginx) & Принимать запросы клиента, обеспечивать работу HTTPS, предоставлять статические файлы фронтэнд-части веб-приложения \\\hline
    Database Server (PostgreSQL) & Хранить данные, которые должны храниться длительное время \\\hline
    RabbitMQ & Обеспечивать обмен сообщениями между компонентами веб-приложения \\\hline
    Application Server & Обрабатывать запросы пользователя \\\hline
    Translation consumer & Переводить статьи при помощи внешнего сервиса \\\hline
    Mailing consumer & Отправлять электронные письма при помощи внешнего сервиса \\\hline
    Redis & Хранить данные с маленьким сроком жизни, выступать транспортом для отправки уведомлений о завершении перевода статей \\\hline
    GPT provider & Переводить тексты по запросу \\\hline
    Mailing service & Отправлять электронные письма по запросу \\\hline
    Client (Vivaldi) & Отображать фронтэнд-часть веб-приложения, отправлять запросы пользователя, отображать ответы сервера \\\hline
\end{longtable}

Таким образом, веб-приложение состоит из различных компонентов, каждый из которых выполняет собственные функции.

\SubsectionBetweenParagraphs{Выводы}

\begin{enumerate}
    \item Была рассмотрена функциональность веб-приложения «GPTranslate» для всех ролей: гостя, пользователя, модератора и администратора. Гостям доступна регистрация и аутентификация. Пользователи могут загружать статьи, переводить их, а также управлять своими конфигурациями переводчика и оставлять жалобы на переведённые статьи. Модераторы рассматривают жалобы, а администраторы могут управлять списками пользователей, моделей и стилей перевода, а также получать статистику жалоб по моделям и стилям перевода. Общее количество функций веб-приложения составляет 22.
    \item Рассмотрена логическая схема базы данных веб-приложения, которая включает 13 таблиц. Таблицы хранят данные о пользователях, статьях, конфигурациях и других.
    \item Рассмотрена архитектура веб-приложения. Использование RabbitMQ позволяет сервисам передавать сообщения между собой, Redis позволяет хранить данные с малым сроком жизни, а PostgreSQL – долговременные данные.
\end{enumerate}
