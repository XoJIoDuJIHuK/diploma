Первый параграф начинается здесь. Когда он закончится, 
следующий блок – это формула, и между ним и предыдущим 
абзацем должно быть ровно 14 pt (заботится equation).

\begin{equation}\label{eq:sum_expenses}
  \text{С}_\text{р} 
    = \text{С}_\text{оз} 
    + \text{С}_\text{дз} 
    + \text{С}_\text{фсзн} 
    + \text{С}_\text{бгс} 
    + \text{С}_\text{пз} 
    + \text{С}_\text{обп, обх}.
\end{equation}
% − Здесь abovedisplayskip=14pt дал 14pt между этим и предыдущим абзацем.
% − Сразу после формулы belowdisplayskip=14pt даст 14pt вниз.

\begin{symblock}
  \begin{itemize}[nosep,
                   leftmargin=0pt,
                   labelindent=0pt,
                   topsep=0pt,
                   partopsep=0pt,
                   itemsep=0pt,
                   parsep=0pt]
    \item[] где $\text{С}_\text{р}$ – итоговая себестоимость, руб.;
    \item[] где $\text{С}_\text{оз}$ – основная зарплата, руб.;
    \item[] где $\text{С}_\text{дз}$ – доп. выплаты, руб.;
    % … и так далее
  \end{itemize}
\end{symblock}
% − symblock сверху “тихо” (уже было 14pt снизу у equation), 
%   а в конце symblock вставляет \vspace*{14pt}.

\begin{gather*}
  \text{С}_\text{р}
    = 16171.92 + 2425.79 + 6323.22 + 111.59 + 148 + 6885.96
    = 33266.47\,\text{руб.}
\end{gather*}
% − Перед gather* сработал \AtBeginEnvironment{gather*} → abovedisplayskip=0pt,
%   то есть никакого « дополнительного » отступа сверху. 
%   Но symblock уже вставил ровно 14pt, и больше ничего не прибавилось.
% − После gather* belowdisplayskip=14pt сам по себе вставляет 14pt вниз.

Следующий абзац обычного текста — и между ним и gather* ровно 14 pt 
(этого “ниже” в gather* заботится belowdisplayskip).

% ========================= Пример 2 =========================
% параграф → формула → параграф

А вот новый параграф.  

\begin{equation}
  a^2 + b^2 = c^2
\end{equation}
% − abovedisplayskip=14pt сработал сверху, ниже = 14pt.  

Абзац после формулы.  % ровно 14 pt между ними (заботится belowdisplayskip).

% ========================= Пример 3 =========================
% параграф → формула → символы → параграф

Новый параграф.  

\begin{equation}
  E = mc^2
\end{equation}

\begin{symblock}
  \begin{itemize}[nosep, leftmargin=0pt, labelindent=0pt,
                   topsep=0pt, partopsep=0pt, 
                   itemsep=0pt, parsep=0pt]
    \item[] $E$ – энергия, Дж;
    \item[] $m$ – масса, кг;
    \item[] $c$ – скорость света, м/с.
  \end{itemize}
\end{symblock}

Новый абзац после символов.  % ровно 14 pt (symblock в конце вставил 14 pt).

% ========================= Пример 4 =========================
% параграф → формула → вычисления → параграф

Следующий абзац.  

\begin{equation}
  F = ma
\end{equation}

\begin{gather*}
  F = 10 \times 5 = 50\,\text{Н}.
\end{gather*}

Абзац далее.  % ровно 14 pt (ниже gather* стоит global belowdisplayskip=14pt
