\documentclass[14pt]{extarticle}

% —————————————————————————————————————————————————————————————
% 1) ПАКЕТЫ И ОСНОВНЫЕ НАСТРОЙКИ
% —————————————————————————————————————————————————————————————
\usepackage{fontspec}
\setmainfont{Times New Roman}

\usepackage[russian,english]{babel}
\selectlanguage{russian}

\usepackage[a4paper,left=30mm,right=15mm,top=20mm,bottom=20mm]{geometry}
\usepackage{setspace}
\singlespacing
\usepackage{indentfirst}
\setlength{\parindent}{12.5mm}
\setlength{\parskip}{0pt}

\usepackage{lua-visual-debug}

\usepackage{amsmath}    % equation, gather*, multline* и пр.
\usepackage{enumitem}   % для itemize с “nosep, topsep=0pt” и т. д.
\usepackage{etoolbox}   % для \AtBeginEnvironment (при желании можно не использовать)

% —————————————————————————————————————————————————————————————
% 2) ОБНУЛЯЕМ ВСЕ СТАНДАРТНЫЕ SKIP'Ы ВОКРУГ DISPLAY-ОКРУЖЕНИЙ
% Это гарантирует, что LaTeX НЕ будет подставлять никакие «автоматические»
% 14–25 pt перед/после equation, gather*, multline* и т. д.
% —————————————————————————————————————————————————————————————
\AtBeginDocument{%
  \setlength{\abovedisplayskip}{0pt}%
  \setlength{\abovedisplayshortskip}{0pt}%
  \setlength{\belowdisplayskip}{0pt}%
  \setlength{\belowdisplayshortskip}{0pt}%
}

% —————————————————————————————————————————————————————————————
% 3) ОПРЕДЕЛЯЕМ «EQBLOCK», «SYMBLOCK» И «CALCBLOCK»
% Эти три окружения сами вставляют по 14pt там, где нужно.
% —————————————————————————————————————————————————————————————
\makeatletter

% eqblock: ровно 14pt сверху И 14pt снизу
\newenvironment{eqblock}{%
  \par\vspace*{14pt}\noindent  % 14pt перед формулой
}{%
  \vspace*{14pt}\par          % 14pt после формулы
}

% symblock: 0pt сверху (т. к. eqblock концу уже дал 14pt), 14pt снизу
\newenvironment{symblock}{%
  \par\noindent  % 0pt сверху, предыдущий eqblock уже вставил 14pt внизу
}{%
  \vspace*{14pt}\par  % 14pt после списка символов
}

% calcblock: 0pt сверху (т. к. symblock концу уже дал 14pt), 14pt снизу
\newenvironment{calcblock}{%
  \par\noindent  % 0pt сверху
}{%
  \vspace*{14pt}\par  % 14pt после блока вычислений
}

\makeatother

% —————————————————————————————————————————————————————————————
% 4) РЕКОМЕНДУЕМОЕ ОКРУЖЕНИЕ ДЛЯ ДЛИННЫХ ВЫЧИСЛЕНИЙ (чтобы автоматом было
%    перенесение по ширине страницы). Если краткая запись, можно и gather*,
%    но тогда придётся вручную ставить \\ там, где нужно переносить.
% —————————————————————————————————————————————————————————————
% Здесь показываем пример с multline*
% (ниже в тексте вы его увидите в действии).

% —————————————————————————————————————————————————————————————
% БОЛЬШЕ НИЧЕГО МЕНЯТЬ НЕ НУЖНО — дальше идёт тело документа
% —————————————————————————————————————————————————————————————

\begin{document}

% =========================================
% ПРИМЕР 1: Параграф → Формула → Символы → Вычисления → Параграф
% =========================================

Первый параграф начинается здесь. Когда он закончится, 
между ним и формулой должно быть ровно 14 pt (eqblock сверху).

\begin{eqblock}
  \begin{equation}\label{eq:sum_expenses}
    \text{С}_\text{р} 
      = \text{С}_\text{оз} 
      + \text{С}_\text{дз} 
      + \text{С}_\text{фсзн} 
      + \text{С}_\text{бгс} 
      + \text{С}_\text{пз} 
      + \text{С}_\text{обп, обх}.
  \end{equation}
\end{eqblock}
% — eqblock сверху вставил 14pt между предыдущим абзацем и формулой,
%   eqblock снизу вставит 14pt между формулой и списком символов.

\begin{symblock}
  \begin{itemize}[nosep,
                   leftmargin=0pt,
                   labelindent=0pt,
                   topsep=0pt,
                   partopsep=0pt,
                   itemsep=0pt,
                   parsep=0pt]
    \item[] где $\text{С}_\text{р}$  – итоговая себестоимость, руб.;
    \item[] где $\text{С}_\text{оз}$ – основная зарплата, руб.;
    \item[] где $\text{С}_\text{дз}$ – доп. выплаты, руб.;
    % … остальные пункты …
  \end{itemize}
\end{symblock}
% — symblock снизу вставит 14pt между списком символов и следующим блоком.

% ДЛИННЫЕ ВЫЧИСЛЕНИЯ:
\begin{calcblock}
  \begin{multline*}
    \text{С}_\text{р} = 16171.92 
      + 2425.79 
      + 6323.22 \\
      + 111.59 
      + 148 
      + 6885.96 
      = 33266.47\,\text{руб.}
  \end{multline*}
\end{calcblock}
% — calcblock снизу вставит 14pt между вычислениями и следующим абзацем.

Следующий абзац обычного текста.  % ровно 14 pt от calcblock.

% =========================================
% ПРИМЕР 2: Параграф → Формула → Параграф (без блока символов и вычислений)
% =========================================

Новый абзац перед формулой.  

\begin{eqblock}
  \begin{equation}
    a^2 + b^2 = c^2
  \end{equation}
\end{eqblock}

Следующий абзац.  % ровно 14 pt между формулой и этим абзацем (eqblock снизу).

% =========================================
% ПРИМЕР 3: Параграф → Формула → Символы → Параграф (без вычислений)
% =========================================

Новый параграф.  

\begin{eqblock}
  \begin{equation}
    E = mc^2
  \end{equation}
\end{eqblock}

\begin{symblock}
  \begin{itemize}[nosep, leftmargin=0pt, labelindent=0pt,
                   topsep=0pt, partopsep=0pt, itemsep=0pt, parsep=0pt]
    \item[] $E$ – энергия, Дж;
    \item[] $m$ – масса, кг;
    \item[] $c$ – скорость света, м/с.
  \end{itemize}
\end{symblock}

Новый абзац после символов.  % ровно 14 pt (symblock снизу).

% =========================================
% ПРИМЕР 4: Параграф → Формула → Вычисления → Параграф (без символов)
% =========================================

Следующий абзац.  

\begin{eqblock}
  \begin{equation}
    F = ma
  \end{equation}
\end{eqblock}

\begin{calcblock}
  \begin{multline*}
    F = 10 \times 5 
      = 50\,\text{Н}.
  \end{multline*}
\end{calcblock}

Абзац далее.  % ровно 14 pt (calcblock снизу).

\end{document}
